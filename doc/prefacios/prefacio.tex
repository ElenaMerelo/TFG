\thispagestyle{empty}

\begin{center}
{\large\bfseries Análisis de redes causales en deportes de equipo }\\
\end{center}
\begin{center}
Elena Merelo Molina\\
\end{center}

%\vspace{0.7cm}

\vspace{0.5cm}
\noindent{\textbf{Palabras clave}: \textit{software libre, redes bayesianas, redes causales, 
entropía, entropía de Tsallis, inferencia estadística, teorema de Bayes}}
\vspace{0.7cm}

\noindent{\textbf{Resumen}\\
En la parte informática, se creará una bibilioteca para poder analizar de forma causal redes de pases de un 
equipo tanto a lo largo de un partido como a lo largo de una competición, y poder visualizar tanto el 
desempeño en el campo como su evolución, vinculándolo al desempeño de jugadoras específicas, pudiendo 
calificarlas de esta forma más allá de las puras medidas reticulares.

Los deportes de equipo, especialmente el fútbol, han sido analizados repetidamente desde el punto de vista 
de las redes complejas, buscando correlación entre meso y macroestructuras reticulares y el rendimiento del 
equipo. Sin embargo, no se ha hecho ningún análisis con redes causales, buscando relaciones causa-efecto en 
las redes de pases y, una vez más, su influencia en los resultados que se obtienen a lo largo de una temporada 
o su capacidad para encontrar descriptores macro del equipo a lo largo de una competición determinada, 
dependiendo o no del rival que haya enfrente.
}
\cleardoublepage

\begin{center}
	{\large\bfseries Causal networks analysis in team sports}\\
\end{center}
\begin{center}
	Elena Merelo Molina\\
\end{center}
\vspace{0.5cm}
\noindent{\textbf{Keywords}: \textit{open source}, \textit{bayesian networks}, \textit{causal networks}, 
\textit{entropy}, \textit{Tsallis entropy}, \textit{inference}, \textit{statistics}, \textit{Bayes theorem}
}
\vspace{0.7cm}

\noindent{\textbf{Abstract}\\
A library will be created to be able to causally analyze passes networks of a
team both throughout a match and throughout a competition, and to be able to visualize both the
performance on the field and its evolution, linking it to the performance of specific players, thus being able to
qualify them in this way beyond the pure reticular measures.

Team sports, especially football, have been repeatedly analyzed from the point of view of
of complex networks, looking for correlations between reticular meso and macrostructures and the performance of the
team. However, no analysis has been done with causal networks, looking for cause-effect relationships in
passing networks and, once again, their influence on the results obtained throughout a season
or its ability to find team macro descriptors throughout a given competition,
depending or not on the rival.}
\cleardoublepage

\thispagestyle{empty}

\noindent\rule[-1ex]{\textwidth}{2pt}\\[4.5ex]

D. \textbf{Juan Julián Merelo Guervós}, Profesor del departamento de ATC, y D. \textbf{Úrsula Torres Parejo}, 
Profesora del Departamento de Estadística e Investigación Operativa

\vspace{0.5cm}

\textbf{Informo:}

\vspace{0.5cm}

Que el presente trabajo, titulado \textit{\textbf{Análisis de redes causales en deportes de equipo}},
ha sido realizado bajo nuestra supervisión por \textbf{Elena Merelo Molina}, y autorizamos la defensa de dicho 
trabajo ante el tribunal que corresponda.

\vspace{0.5cm}

Y para que conste, expiden y firman el presente informe en Granada a 31 de septiembre de 2022.

\vspace{1cm}

\textbf{El/la director(a)/es: }

\vspace{5cm}

\noindent \textbf{(Juan Julián Merelo Guervós, Úrsula Torres Parejo)}

\chapter*{Agradecimientos} \label{agradec}

A mis padres, por creer en mí cuando yo no lo hago, animarme a continuar las innumerables veces que tiro la 
toalla y ser tan fundamentales en mi día a día, darle un toque especial y hacerlo mucho mejor todo. Por estar 
siempre ahí, haberme dado y aportado tantísimo en mi vida. A mis hermanas, por apoyarme y aceptarme, y hacer los 
días más entretenidos. A la tienda Alehop, porque sin el ventilador que les compramos habría sido mucho 
más pesado trabajar durante esta horrible ola de calor contínua que llamamos verano en Granada. Al Mercadona, 
por vender chocolate y queso tan rico. A panadería Geni (kiosko de la plaza Mariana Pineda), por hacer que año tras 
año me levante con ilusión al saber que luego me tomaré una tostada del mejor pan de Granada, y porque su 
amabilidad, simpatía y consideración hacen que esperar las colas que suele tener sea más leve. A las personas 
que han aparecido aquí y allí, o me han acompañado y ayudado desde el principio en esta carrera de fondo, 
iluminando el camino por este túnel larguísimo que es el doble grado, y que más veces de las que no se 
sentía más cueva o pozo que otra cosa. 


