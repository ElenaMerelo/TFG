\thispagestyle{empty}

\begin{center}
{\large\bfseries Análisis de redes causales en deportes de equipo }\\
\end{center}
\begin{center}
Elena Merelo Molina\\
\end{center}

%\vspace{0.7cm}

\vspace{0.5cm}
\noindent{\textbf{Palabras clave}: \textit{software libre, redes bayesianas, redes causales, 
entropía, entropía de Tsallis, inferencia estadística, teorema de Bayes}}
\vspace{0.7cm}

\noindent{\textbf{Resumen}\\
	
}
\cleardoublepage

\begin{center}
	{\large\bfseries Same, but in English}\\
\end{center}
\begin{center}
	Elena Merelo Molina\\
\end{center}
\vspace{0.5cm}
\noindent{\textbf{Keywords}: \textit{open source}, \textit{bayesian networks}, \textit{causal networks}, 
\textit{entropy}, \textit{Tsallis entropy}, \textit{inference}, \textit{statistics}, \textit{Bayes theorem}
}
\vspace{0.7cm}

\noindent{\textbf{Abstract}\\

}
\cleardoublepage

\thispagestyle{empty}

\noindent\rule[-1ex]{\textwidth}{2pt}\\[4.5ex]

D. \textbf{Juan Julián Merelo Guervós}, Profesor del departamento de ATC, y D. \textbf{Úrsula Torres Parejo}, 
Profesora del Departamento de Estadística e Investigación Operativa

\vspace{0.5cm}

\textbf{Informo:}

\vspace{0.5cm}

Que el presente trabajo, titulado \textit{\textbf{Análisis de redes causales en deportes de equipo}},
ha sido realizado bajo mi supervisión por \textbf{Elena Merelo Molina}, y autorizo la defensa de dicho 
trabajo ante el tribunal que corresponda.

\vspace{0.5cm}

Y para que conste, expiden y firman el presente informe en Granada a 31 de septiembre de 2022.

\vspace{1cm}

\textbf{El/la director(a)/es: }

\vspace{5cm}

\noindent \textbf{(Juan Julián Merelo Guervós, Úrsula Torres Parejo)}

\chapter*{Agradecimientos}

A mis padres, por creer en mí cuando yo no lo hago, animarme a continuar las innumerables veces que tiro la 
toalla y ser tan fundamentales en mi día a día, darle un toque especial y hacerlo mucho mejor todo. Por estar 
siempre ahí, haberme dado y aportado tantísimo en mi vida. A mis hermanas, por apoyarme y aceptarme, y hacer los 
días más entretenidos. A la tienda Alehop, porque sin el ventilador que les compramos habría sido mucho 
más pesado trabajar durante esta horrible ola de calor contínua que llamamos verano en Granada. Al Mercadona, 
por vender chocolate y queso tan rico. A panadería Geni, por hacer que me levante con ilusión al saber que 
luego me tomaré una tostada del mejor pan de Granada, y porque su amabilidad, simpatía y consideración 
hacen que esperar las colas que suele tener sea más leve. A las personas que han aparecido aquí y allí, o 
me han acompañado y ayudado desde el principio en esta carrera de fondo, iluminando el camino por este 
túnel larguísimo que es el doble grado, y que más veces de las que no se sentía más cueva o pozo que otra cosa. 


