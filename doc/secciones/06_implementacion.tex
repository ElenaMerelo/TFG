\chapter{Implementación y resultados}
En este capítulo presentaremos los resultados experimentales, y qué nos llevó a ellos, siguiendo 
la metodología ya descrita en \ref{ch:metod}.

\section{Resultados}
Como caso de estudio, nos centraremos en el equipo femenino de Noruega, dejando para trabajos futuros el resto 
de análisis y preguntas que nos podemos plantear.

Más específicamente, usaremos como fuente la 
\href{https://es.uefa.com/womenseuro/news/0258-0e223de64a73-41a64bf308f9-1000--todos-los-resultados/?iv=true}{EURO 
Femenina de la UEFA 2022}, en la cual Inglaterra resultó ganadora, y Noruega perdió 
\href{https://es.uefa.com/womenseuro/match/2032209--england-vs-norway/}{un partido} contra ellas por ocho 
goles a cero (es de notar que seis fueron tan solo en la primera parte), constituyendo uno de los márgenes más 
grandes de victoria (en las fases finales) de la historia 
del campeonato, además de uno de los partidos (nuevamente, de los finales) en los que más goles se han marcado. 
Nuestro objetivo será estudiar por qué ocurrió esto, si podemos verlo reflejado de alguna manera 
en las redes de pases o entropía de ambos equipos. Añadir que luego tampoco pasaron a cuartos de final, al perder 
contra Austria por \href{https://es.uefa.com/womenseuro/match/2032211--austria-vs-norway/}{un gol}. Será interesante 
comparar más a fondo el desempeño de Noruega a lo largo de este campeonato, del cual (para nuestra agradable sorpresa) se 
\href{https://statsbomb.com/articles/soccer/statsbomb-release-free-360-data-womens-euro-2022-available-now/}{
liberaron los datos} el cuatro de agosto, por \href{https://statsbomb.com/}{StatsBomb}, una empresa de 
análisis de fútbol y visualización de datos con sede en el Reino Unido.

Esa derrota fue sorprendente teniendo en cuenta que, si consultamos 
\href{https://www.uefa.com/womenseuro/news/023d-0e16a7c86b1c-05ff0a6fb380-1000--uefa-women-s-euro-facts-and-figures-player-records-most-goals-b/}{
el ``cuadro de honor de la EURO femenina''}, vemos que previamente Noruega ha ganado dos veces esta 
competición, contra una vez que ya ganó Inglaterra; han estado seis veces en la final versus tres que ha 
estado Inglaterra, y nueve veces en semifinales, contra seis de Inglaterra. 
Si consultamos las \href{https://www.uefa.com/womenseuro/statistics/}{estadísticas por equipo y jugadoras}, sin 
duda Inglaterra y Alemania son las que se imponen, apareciendo Noruega únicamente en un quinto puesto en cuanto 
a posesión del balón (52.7 \%, contra un máximo de 63.8\% por parte de España), un sexto puesto por una 
precisión de pases del 79.7\% (contra máximo de un 86.8\%, también de España), y otro quinto puesto por catorce paradas 
de la portera, en comparación con el primer puesto por veintitrés salvadas de Holanda. 

Nuestra pregunta es pues ¿qué les pasó a las noruegas este año? El entrenador (ahora ex-entrenador) Martin Sjögren 
\href{https://www.nrk.no/sport/massiv-kritikk-mot-sjogren-etter-tidenes-storste-norge-tap-1.16034807}{fue muy 
criticado} por no hacer cambios de formación o jugadores en esos primeros 47 minutos en los que volaron goles, 
y al ser ese partido de la EURO 2022 contra Inglaterra la mayor derrota de Noruega en la historia; anteriormente 
habían perdido 0-4 contra Alemania en la Eurocopa de 2009, 0-5 contra China en el mundial de 1999, o 
\href{https://www.nrk.no/sport/norges-landslag-knust-av-nederland--1.15538927}{un 0-7} contra Holanda el año pasado.
\href{https://www.nrk.no/sport/norge-mareritt-mot-england-i-em_-_-dette-er-direkte-pinlig-1.16034607}{Se culpa} a la 
falta de estructura defensiva y presión ofensiva; alegan que empezaron bien, mas una vez Inglaterra marcó el primer gol en 
el minuto 12 no pudieron pararles, y fueron cuesta abajo. Veremos si esto se refleja en los datos obtenidos y por qué 
salió tan mal, considerando en último lugar que Noruega tiene 
\href{https://www.nrk.no/sport/engelsk-forbauselse_-_-jeg-kan-ikke-helt-tro-det-jeg-nettopp-har-vaert-vitne-til-1.16034919}{
jugadoras de los mejores equipos del mundo}. 

No obstante, desde que en Inglaterra tienen a la entrenadora Sarina Wiegman, no 
han perdido los últimos dieciséis partidos. ¿Tan grande es la influencia del entrenador? 
\href{https://www.nrk.no/sport/hvis-det-er-en-plan_-er-den-usynlig-1.16038718}{En este artículo} ponen en evidencia los problemas 
que se han ido arrastrando desde que Sjögren es entrenador, con un rendimiento lejos del esperado (teniendo en 
cuenta la buena prestación de las jugadoras cuando juegan en los equipos en que están fichadas) en las últimas 
competiciones. Se discute que como equipo cometieron errores a la hora de posicionarse, dejaron libre la zona de 
defensa y jugadoras que podrían haber sido decisivas se quedaron aisladas, con lo que abrieron las puertas a 
las inglesas a dominar el juego. 

\href{https://www.nrk.no/sport/norsk-fiasko-i-em_-_-en-varslet-katastrofe-1.16039578}{En el partido contra Austria},
que debían ganar, no hubo un solo disparo a puerta antes del minuto 88, por lo que siguieron siendo una sombra de sí 
mismas, pese a que, en sus palabras, usaron los días posteriores para reconstruir su orgullo y autoestima, y recomponerse 
de tal golpe bajo. No consiguieron estar a la altura, dejar de ser una sombra de ellas mismas, y hubieron de pasar 
63 minutos antes de que el entrenador hiciera un cambio (el gol de Austria fue en el minuto 37). Se cometieron también 
errores aquí y allí, pero está lejos de ser el partido contra Inglaterra. ¿Podremos ver esta diferencia reflejada 
en las redes de pases o entropía del equipo? Principalmente, ¿es problema de la formación 
(en \href{https://www.nrk.no/sport/reiten-ut-mot-sjogren-grep_-_-pa-tide-med-noko-nytt-1.16085631}{la nueva 
entrenadora Hege Riise} afirma que habría sido mejor jugar 4-3-3, en vez de 4-2-3-1), del entrenador, de las 
jugadoras que no supieron coordinarse, o simplemente no están a la altura de Inglaterra? Con este trabajo intentaremos 
encontrar respuestas.

\section{Elección de fuentes de datos}
% issue #80, #71

\section{Paquetes de R}
% issue #79

\section{Diseño experimental}
En este apartado particularizaremos nuestro estudio a los resultados que acabamos de mencionar. 