\chapter{Conclusiones y trabajos futuros}
En este trabajo hemos intentado modelar la manera en que los equipos consiguen, en mayor o menor medida,
alcanzar un equilibrio entre organización y desorganización. Hemos visto reflejado el hecho de que una 
mayor entropía implica una posición más baja en un campeonato, y en trabajos futuros nos gustaría estudiarlo 
más en profundidad. En este apartado veremos específicamente las conclusiones a las que hemos llegado a partir 
de este estudio preliminar, y los caminos que se pueden seguir a partir de aquí.

Hemos de tener en mente que por un lado, un equipo ha de intentar minimizar su entropía para poder maximizar 
la comunicación entre sus jugadores, a la vez que por el otro, deben maximizar su entropía para maximizar sus 
grados de libertad y evitar que el rival identifique un patrón ordenado de comportamiento.
\cite{entropy-analysis}

El desempeño de un equipo de fútbol está claramente influenciado por el equipo contra el que juegan, así que para 
trabajos futuros dejamos lo de analizar, visualizar y estudiar el partido en el que se enfrentaron Inglaterra 
y Noruega. \cite{donogue} teorizó sobre la naturaleza adaptativa de un jugador, pero no midió el desempeño de los 
jugadores en términos de entropía. Nosotros hemos estudiado el rendimiento de un equipo de fútbol en base 
a la entropía de los pases a lo largo de una competición, y queda ver cómo esta medida se adapta al equipo 
rival, o comparar cómo varía de partido a partido.

Otra cuestión importante que no podemos dejar de lado es que la correlación entre el buen desempeño de un 
equipo y la entropía que se obtiene de sus redes de pases puede tener su base en la posesión del balón 
del mismo; cuánto más tiempo lo tengan, más pases harán, luego más se pueden organizar y son los que 
dominan el juego, pero puede salir una entropía más alta, que no es que indique más caos o desorden, 
sino que hay más grados de libertad, y no tiene por qué significar que vayan a tener una posición más baja en la 
competición. Sería interesante pues ampliar este estudio que hemos hecho 
incluyendo los datos correspondientes a la posesión del balón de cada equipo en la tabla de distribuciones 
condicionadas de la red bayesiana, o como causalidad. En todo caso, introducir introducir redes bayesianas de las 
maneras que estudiábamos en el apartado de análisis puede ser muy esclarecedor y ampliar lo que se obtiene, y lo 
dejamos para próximos trabajos.