\chapter{Introducción}

En este capítulo se describen la motivación del proyecto y los objetivos que
se quieren alcanzar durante el desarrollo, así como una introducción histórica al
a la aplicación del análisis al fútbol. 

\section{Motivación}

Los deportes de equipo, especialmente el fútbol, han sido analizados 
repetidamente desde el punto de vista de las redes complejas, buscando 
correlación entre meso y macroestructuras reticulares y el desempeño del 
equipo. Sin embargo, no se ha hecho ningún análisis con redes causales, 
buscando relaciones causa-efecto en las redes de pases y, una vez más, 
su influencia en el desempeño del equipo o su capacidad para encontrar 
descriptores macro del equipo a lo largo de una competición determinada, 
dependiendo o no del rival que haya enfrente.

\section{Objetivos del trabajo} \label{sect:goals}

Una vez que el problema que se quiere resolver está claro, lo que perseguimos
con este proyecto es: 

\begin{enumerate}
    \item \label{obj:1} Aprender a desarrollar desde cero un producto software de calidad, empleando desarrollo ágil.
    \item \label{obj:2} Aplicar inferencia causal al análisis de partidos de fútbol.
    \item \label{obj:3} Crear una biblioteca que pueda usar el analista del equipo deportivo de un equipo.
\end{enumerate}

\section{Historia de la analítica en el fútbol}

Todo empieza con Charles Reep, antiguo comandante de la Royal Air Force, quién
recopilaba datos jugada por jugada y fue consultor de equipos de la Liga de 
Fútbol ya en los años cincuenta. Su investigación fue bastante innovadora para 
su tiempo, incluso aunque fuera errónea. Reep reunía datos sobre la frecuencia 
con la que un número de pases exitosos se encadenaban juntos, y cuán común era 
que resultaran goles de ello. Asimismo, determinó que la probabilidad de un 
equipo de mantener la posesión del balón caía precipitadamente con cada intento 
de pase, y que la mayoría de los goles eran marcados cuando se hacían menos de 
tres pases, los cuales a menudo se originan en contraataques rápidos.

Para Reep, esto significaba que los equipos deberían dejar de intentar controlar la posesión,
y de hacerse paso a través de la defensa con pases y pases. En cambio,
debían enfocarse en llevar la pelota campo abajo en la menor cantidad de movimientos posibles
en la ofensiva, y aplicando presión en la defensa para generar
contraataques. Los números parecían sugerir que el juego largo era el
táctica más eficiente para el éxito futbolístico.

But subsequent analysis has discredited this way of thinking. Reep’s mistake 
was to fixate on the percentage of goals generated by passing sequences of 
various lengths. Instead, he should have flipped things around, focusing on 
the probability that a given sequence would produce a goal. Yes, a large 
proportion of goals are generated on short possessions, but soccer is also 
fundamentally a game of short possessions and frequent turnovers. If you 
account for how often each sequence-length occurs during the flow of play, 
of course more goals are going to come off of smaller sequences — after all, 
they’re easily the most common type of sequence. But that doesn’t mean a 
small sequence has a higher probability of leading to a goal.

To the contrary, a team’s probability of scoring goes up as it strings together more successful passes. The implication of this statistical about-face is that maintaining possession is important in soccer.2 There’s a good relationship3 between a team’s time spent in control of the ball and its ability to generate shots on target, which in turn is hugely predictive of a team’s scoring rate and, consequently, its placement in the league table. While there’s less rhyme or reason to the rate at which teams convert those scoring chances into goals, modern analysis has ascertained that possession plays a big role in creating offensive opportunities, and that effective short passing — fueled largely by having pass targets move to soft spots in the defense before ever receiving the ball — is strongly associated with building and maintaining possession.

As for the long ball, it’s proven futile in today’s game. During the 2013-14 English Premier League season, the percentage of a team’s passes classified as “long” by Whoscored.com’s data was very negatively correlated4 with how many goals it scored.5
The same goes for trying to spearhead an offense from the wings instead of attacking up the middle. In their book, Anderson and Sally write about a seminal piece of quantitative analysis on the 1986 World Cup from researcher Mike Hughes: “Successful teams played a passing game through the middle in their own half and approached the other end of the pitch predominantly in the central areas of the field, while the unsuccessful teams played significantly more to the wings.” The numbers from the 2013-14 season in Europe’s “Big Four” leagues6 bear this out as well. The percentage of a team’s attacks made up the middle did have a moderately positive7 relationship to its scoring rate relative to the league average, while the relationship between wing attacks and scoring was of the same magnitude and in the negative direction.

This, coupled with the fact that corner kicks are surprisingly ineffective at generating goals, is probably related to the negative correlation between a team’s propensity for winning aerial duels8 and its overall goal-scoring rate. By the numbers, it’s a losing bet to count on goals in the air via set pieces — or even off crosses in open play — as a steady way to generate offense, just as it is to rely on the long ball to consistently produce chances. Instead, the statistics seem to support an approach more in line with the artful tiki-taka style exemplified most notably by FC Barcelona and the Spanish national team. In soccer, data and aesthetics are not mutually exclusive, just as they aren’t in any other sport.

That’s the one bit of analytics wisdom that could stand to become more conventional. For now, though, we have a reasonably good idea of which metrics correlate with a team’s success more than others. Keep those in mind as you gorge on soccer over the next month.

Not much is known about the use of statistics in soccer before 1996. Older fans know that Arsene Wenger used a computer program during his time managing Monaco in the late 1980s. In any case, at this time clubs still relied heavily on their own data collection and storage methods.

In 1996, Opta began collecting match data for the English Premier League. This included all of the stats that a casual soccer fan is used to – number of passes, number of tackles, distance travelled, etc. In many ways Opta’s collection of this data was the starting point of what we now refer to as soccer analytics.

The use of stats was also headlined in 1999 when Bolton hired “Big Sam” Allardyce as their new manager. Legend has it Big Sam realized he couldn’t match the payrolls of bigger clubs and instead decided to hire a team of statisticians. They observed that the ball changed hands about 400 times in a single game, and that once a team lost the ball, quickly getting back into a solid defensive position was one of the best ways to avoid conceding goals. In addition, it is rumoured that Big Sam’s (often-criticized) direct approach is a result of the observation that about 30% of goals come from set pieces. By focusing more on set pieces, he was able to have Bolton score about 50% of their goals this way.

In 2003 Michael Lewis released Moneyball and its influence quickly spread across the Atlantic. The book’s release naturally started raising questions about the use of data in soccer. A couple of transfer market irrationalities were quickly exposed, most notably the salary/career length ratio for keepers and strikers and a strange bias towards recruiting taller players.

Fuelled by stats like Opta’s and the ideas of efficiency from baseball, analytics work from roughly 2003 until now has exposed many limitations in the popular stats. At the team level, a surprisingly low correlation was found between win percentage and metrics like the number of shots, possession, or number of passes. To this day, shot totals are used as some kind of proof that Team A played better than Team B. This however does little to explain the efficient tactics of clubs like Atletico Madrid, which take pride in limiting their opponents to a barrage of hopeful shots from outside the box while Atletico themselves focus on creating a handful of very good chances in the danger zone. At the player level, stats such as the number of tackles or distance travelled also didn’t seem to be very indicative of player quality. Nonetheless most of these are still used today in both TV broadcasts and journalistic work.


Football Analytics in a nutshell
Football analytics is basically data science principles applied to football.

The sports industry has begun to develop towards data like any other industry at the beginning of the XXI century.

It started in the US with baseball, American football, and basketball. Now every sport has at least some fans who try to operate data to understand their sports passion.

Clubs, players, competition stakeholders, and betting companies are leaning to data to drive their decisions.

The first objective of this transformation is to analyze and model the game in the most objective way. To improve sports performances.

Understand players' choices, forecast team performance trends, study tactic and rules evolution, forecast injuries, optimize transfer decisions, find valuable — underrated — players, etc…

The list is infinite: the variety of data and the freedom of football make a perfect playground for data analytics.

Behind the focus on sports performance is the challenge to make data-driven decisions regarding financial decisions.

Come to mind player transfers: it’s the resource with the most leverage.

But in the long term stakeholders would like to improve business assets such as overall revenue, marketing and technology development, staff investments, or stadium rentability.

What data? Where?
Like any data-science scheme, the force of the war is data. Some companies emerged in the last decades to collect, transform and analyze data from football games. Clients for these data are varied:

Football clubs: to improve performances and drive decisions.
Betting companies: to polish their odds and offer more and more options for betters.
Newsrooms: to help readers understand the game. Find an explanation of player performances. Provide deeper analysis.
All these people have different needs. Therefore, it exists mainly three kinds of football data.

Contextual data: main events like goals, replacement, yellow cards, etc… It’s characterized by the minute of the event, so quite basic. Still useful for news and betting companies for live broadcasts on the web.
Event data: much detailed, they track every ball action: passes, tackles, crosses, dribbles, shoots, etc… They are characterized by coordinates data (sometimes in three dimensions), so very useful for any person looking to analyze and create models of the game. Used by football clubs, betting companies, and sometimes newsrooms. Gathered by hand and algorithms.
Tracking data: every player's position and ball position at really low framerates sometimes 15 frames by seconds. Less mature but used by big clubs and researchers. Gathered either by physical sensors or software algorithms.
It’s often too complex and too expensive for clubs, betting companies, and newsrooms to gather data by themself. So a landscape of providers emerged too, with diverse offers, diverse data :

Stats Perform (ex Opta): the big one. It’s one of the first which started to gather football data at scale. They have a range of clients from top football clubs to big newsrooms. They are well known for both contextual and event data.
Statsbomb: the serious challenger. They provide detailed event data feeds and great analytics resources.
Skillcorner/Second Spectrum/Metrica: focused on tracking data and analysis. They use state-of-the-art algorithms or sensors to track all the data they can. The utilization of these data is infinite but complex.



Los deportes de equipo, especialmente el fútbol, han sido analizados 
repetidamente desde el punto de vista de las redes complejas, buscando 
correlación entre meso y macroestructuras reticulares y el desempeño del 
equipo. Sin embargo, no se ha hecho ningún análisis con redes causales, 
buscando relaciones causa-efecto en las redes de pases y, una vez más, 
su influencia en el desempeño del equipo o su capacidad para encontrar 
descriptores macro del equipo a lo largo de una competición determinada, 
dependiendo o no del rival que haya enfrente.

The best reason to try to measure the sport is the same reason people used to say it couldn’t be measured: soccer is hard. Even coaches and analysts and scouts who’ve spent their lives learning to watch it won’t see games quite the same way. There are too many moving parts, too many possibilities to hold in your head at once. Had we but world enough and time, you might rewatch each match over and over to pause and study it and it’d still be impossible to see and remember it all. And if you have to do that for an opponent’s entire season, or a continent you’re charged with scouting? Pretty soon it starts to look like the nerds might be onto something.

Data scales. It doesn’t sleep or steal your food out of the office fridge. It’s objective-ish and consistent-ish, depending on what exactly it measures and how, and honestly sometimes even sketchy data is better than the stories we reduce soccer to without it. The hard part of of analytics comes when we have to boil down the data to stories, too, to make it useful. How do we know what it means—and what we’re missing?

Where do soccer analytics come from?
The story of soccer analytics usually starts with a cautionary tale. At 3:50 p.m. on March 18, 1950—his notes were very precise—a retired Royal Air Force officer named Charles Reep, who had trained as an accountant, began systematically recording events at a Swindon Town match. Reep didn’t just want a detailed record of what happened. Like any good analyst, he wanted to know why it happened and what teams could learn from the data to improve their play. His pet finding over his decades of logging matches was that most goals resulted from possessions of three passes or fewer, which he took to mean that teams should simplify their tactics to get to goal faster, with less of the namby-pamby possession stuff. In an era when Hungary’s “pattern-weaving” passing style was the toast of the world game, Reep was writing articles with headlines like This Pattern-Weaving Talk is All Bunk!

The problem wasn’t Reep’s diligent data collection—it was his analysis. Most possessions are short, especially if you define them strictly enough to ensure that result; if there are a lot more short sequences, it’s no surprise that more goals result from them. A less polemical analyst might have inquired about the rate of goalscoring on possessions of various lengths, or how exactly those three-pass possessions developed. Reep’s own work showed that “60 per cent of all goalscoring moves begin 35 yards from an opponent’s goal,” which might make you wonder whether pattern-weaving was perhaps a good way to get close to goal in the first place. Not Reep, who worked with the great Wolves manager Stan Cullis to implement a style based on the “wholly English” principles of “direct passing.” Wolverhampton’s success in the fifties, including a dramatic upset of the Hungarian champions Honved, was seen as proof of concept and an affirmation of the long ball game. By his own account, Reep gathered data to “provide a counter to reliance upon memory, tradition and personal impressions that led to speculation and soccer ideologies.” But the result was more ideology, now with the false certainty of science.

Contrary to what you’ll sometimes read, Reep wasn’t the first pioneer of soccer analytics. His own interest had been piqued by an account of Herbert Chapman’s statistical approach to coaching Arsenal in the early thirties. Just this year, the internet turned up some stunningly modern-looking data vizzes drawn by hand for individual matches in 1920s Hungary, Reep’s bête noire. A Budapest grad student named Attila Bátorfy discovered that dozens of the charts had run on the front page of a daily newspaper dating back to 1922, and were popular enough to be imitated by sports dailies in Italy and Sweden. It’s not hard to imagine that as long as players have been passing and shooting, there’s been a weirdo standing on the sideline somewhere trying to jot it all down.

These days soccer data collection is big business. Companies like Opta and Statsbomb use human coders and computer vision to log information about every event that happens on the ball: passes, dribbles, shots, headers, fouls, tackles, interceptions, blocks, clearances, claims, and saves. Analysts can count these raw events, divvy them up or plot them out, derive second-order metrics like possession percentages, or use them to build sophisticated models. There’s a growing emphasis on linking data to video, which is undergoing a revolution of its own thanks to global providers like Wyscout. The juiciest stuff on the market is tracking data, which uses cameras or other tech to trace every player’s movements so that analysts can see off-ball patterns and soccer’s most prized commodity, space. You can join tracking data to event data for a more complete picture of what’s happening, but even that won’t tell you which way players are looking or how their bodies are arranged, so some researchers go even further and collect gaze and pose data. Because tracking data is expensive to collect and impossible to come by for, say, youth prospects in the Bolivian second division, analysts sometimes try to extract insights from it and apply them back to event data, which you can buy for just about any league with a TV contract or even a decent camera in the stands following the ball. But even when you’ve got information on what happened in a game, Reep’s story serves as a reminder that the challenge is what you do with it.

What is analytics good for?
I threw this list together last night after a couple beers, and maybe practitioners will have different ideas, but the way I see it the stuff you can measure with game data falls into five categories, from easiest to hardest:

Contribution
Style
Skill
Potential
Big questions
Contribution is the most straightforward, since it’s about measuring who did what (although when you try to put a value on those contributions things get more complicated). Style separates what happened from how and why. Skill, at the player level, tries to give a more nuanced—and speculative—picture of contribution by accounting for circumstances like role, tactics, and team strength. Potential strays even further from what we know into what we wish we knew, trying to project how a team or player or even a particular pattern of play would fare under circumstances different from in the past, like how a young player might grow if he transferred to a team with an unfamiliar playstyle in a tougher league. And then there’s a catchall for research questions that sprawl across or outside the other categories. This might include evaluations of a game model or of specific tactics, like a story I once read about a team of analysts who spent weeks studying their team’s transition patterns so a coach could explain something to players by drawing a single line on a whiteboard.

You can do all of this without data—they’re really just kinds of questions you can ask about soccer, not inherently quantitative problems—and a lot of coaches and sporting directors would rather trust analysts’ eyes than whatever some computer spits out. But the supposed dichotomy between data and video, or data and scouting, or data and knowing anything at all about the game, is transparently phony. Good analytics are always informed by what the nerds call “domain knowledge”—the expertise of players and staff. If you do it right, you might even get some benefits flowing the other direction, too. The best analytics work makes us better at seeing and thinking about the game.

If you’ve heard of sports analytics, you’ve probably also heard of Moneyball, Michael Lewis’s book about how the Oakland A’s front office turned a low-budget baseball team into a powerhouse by using stats to recruit players with undervalued skillsets. The story hooked not only fans but also business travelers killing time in the airport bookshop during an era when an explosion of data had everyone scrambling to extract valuable analytical insights from it. Lewis’s protagonist, Billy Beane, became a cult hero and, I feel pretty confident saying without checking IMDB, the first stats nerd to be played by Brad Pitt.

Not long after Moneyball was published in 2003, Beane was already turning his attention to soccer, starting with Oakland’s neighboring San Jose Earthquakes. But applying analytics to soccer, as people with a penchant for the obvious never get tired of pointing out, is harder than in baseball. There’s more you can potentially do with data in a fluid sport than scout high-OBP catchers or plot an infield shift, but faced with the challenges of getting there, as well as the equally daunting job of convincing decisionmakers that what you’ve found is useful and not another Reepian mistake, not many organizations have been in a hurry to go all in.

Most clubs do at least do some data scouting. The generally accepted best practice is you use stats to pull together a list of some good prospects who fit the profile you’re looking for, doing a much wider and faster first pass than a scouting network could; then you watch a bunch of video to get a more complete picture of the players and narrow things down; and finally you might send out a scout to get to know your favorite guys in their environments and maybe watch a few live games just for kicks. Done right, this can make the recruiting process more efficient and with any luck find you better players for cheaper than the old system of binoculars in the stands and agents hawking their guys on the phone. But while pretty much every club has bought into the idea that they should be doing something with data, you’ll still hear an alarming number of sporting directors brag about scouting a player they loved and then checking his numbers right before the signing, you know, just in case.

While recruiting probably offers the biggest bang for your statistical buck—ask Liverpool, whose elite analytics operation helped them build a squad that became champions of everything on a comparatively reasonable budget—it’s not the only thing data is good for. A lot of clubs also use analytics as part of their opposition scouting. Again, not hard to see why. When Marcelo Bielsa got caught spying on Derby County and responded in the most Bielsa way possible by convening a press conference to lecture reporters on every detail of his weekly game prep, the biggest takeaway from his mountains of binders and endless slides was that his staff was sacrificing sleep to compile information that could have been done at the push of a button. Analyzing a soccer team is hard, and there are definitely parts that are better done with video, but a lot of clubs integrate their data and video operations to make each other’s lives easier. If nothing else, analytics can be the quickest way of pulling up the right clips for analysts to study.

And then there are what you might call the big questions. Billy Beane’s staff was focused on buying cheap wins, but they took their inspiration from Bill James, an idealistic fan prone to writing things like, “I do not start with the numbers any more than a mechanic starts with a monkey wrench. I start with the game, with the things that I see there and the things that people say there. And I ask: Is it true? Can you validate it? Can you measure it? How does it fit with the rest of the machinery?”

The best possible use of analytics isn’t just to measure the game but to understand it. That’s what smart clubs do when they use data to ask questions about how they play, or better yet how they should play. It’s what academic researchers and curious fans do with whatever data they can get their hands on. It’s what Charles Reep and Herbert Chapman and those Hungarian newspaper readers were after, even if it didn’t always work out. When space space space covers analytics, we’ll be asking the same questions James wanted to know—is it true? can you measure it? how does it all fit together?—but about soccer’s more intricate, more beautiful, and altogether extraordinary machine. ❧



Este proyecto es software libre, y está liberado con la licencia \cite{gplv3}.
