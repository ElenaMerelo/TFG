\chapter{Introducción}

En este capítulo se describen la motivación del proyecto y los objetivos que
se quieren alcanzar durante el desarrollo, así como una introducción histórica al
a la aplicación del análisis al fútbol y las redes bayesianas. 

\section{Motivación}

WIP: me queda traducirlo, encajarlo con nuestro trabajo. 

No tenemos que irnos muy atrás en el tiempo para encontrar titulares como 
el siguiente: 'Massiv kritikk mot Sjögren: - Historiens dårligste kampledelse
EM-kampen mot England utviklet seg til å bli tidenes største Norge-tap. Nå kritiseres Martin 
Sjögren for manglende grep fra sidelinjen.', 'Tidenes største Norge-tap: – Dette er direkte pinlig
Norge hadde håp om poeng mot vertsnasjonen England. I stedet ble det en marerittkamp av de sjeldne.', 
o 'Engelsk forbauselse: – Jeg kan ikke helt tro det jeg nettopp har vært vitne til
Englands sjokkresultat mot Norge skaper reaksjoner på balløya. England-trener Sarina Wiegman sier 
hun er overrasket over Norges prestasjon.', 'Norge-kapteinen kvalm etter EM-exit – ekspert vil ha Sjögren sparket
Martin Sjögrens Norge skulle slå tilbake etter flausen sist. Det klarte de ikke, og Norge må allerede nå pakke 
koffertene og reise slukøret hjem fra mesterskapet.', con respecto a lo sucedido 
a Noruega en los campeonatos europeos de fútbol femenino. 

– Det er trist å se på. Jeg synes man har bommet på formasjon og uttak. Det 
er 6-0, det må være mulig å gjøre grep i løpet av disse 47 minuttene, 
kommenterte NRK-ekspert Elise Thorsnes.

TV 2s ekspert Yaw Ihle Amankwah oppsummerte det slik på Twitter:

«Historiens dårligste kampledelse».

Han begrunner det på følgende måte overfor NRK:

– Jeg er på stadion og sitter rett bak Sjögren, og at han tillater 
at den første omgangen spilles på den måten her, uten å ta grep, 
uten å gjøre verken formasjonsendringer eller spilleendringer ... 
Eller egentlig konkret, noe som helst. Det er for meg helt ufattelig. 
Og det er regelrett tjenesteforsømmelse av Sjögren, sier Amankwah til NRK.

Sjögren gjorde ett bytte i pausen, satte inn Guro Bergsvand og tok ut Karina Sævik, og la samtidig om til en fembackslinje. Likevel scoret England to ganger til.

Overfor NRK svarer Sjögren følgende på spørsmål om hvorfor han ikke tok grep tidligere:

– Vi ventet til pausen. Når skal man dra i nødbremsen? Vi gjorde det i pausen, men slapp inn to mål til. Det er usikkert om det hadde stoppet blødningen i første omgang, mener svensken.

På pressekonferansen etter kampen utdyper Sjögren at han vurderte det til at han trengte et stopp for å gjøre endringene.

– Det er alltid enkelt å være etterpåklok. Vi har en halvtime som er ekstremt vanskelig. Kanskje vi skulle gjort noe da, jeg vet ikke om det hadde blitt noe bedre der og da, sier Sjögren.

– Aleksander Schau i NRK-studio sier det er det verste han har sett av et landslag, og opplevde det som en skrekkfilm. Hvordan opplevde du det?

– Det er drittøft å stå der. Og mest av alt er det tøft for spillerne også, man blir et steg etter hele tiden. Så det er klart at det er en tøff sannhet. Nå gjelder det å forsøke og håndtere det og denne kampen, og vise at vi er en sterk gruppe. EM er jo ikke over, det er en stor trøst i at vi fortsatt har alt i våre hender. Og en kamp i nær fremtid mot Østerrike, vinner vi den så går vi videre. Men ja, vi må håndtere denne innsatsen her og den kampen, så kan jeg love at vi skal komme ut og være betydelig bedre.

Jonas Eidevall, som er manager for Arsenals kvinnelag, retter også kritikk mot Sjögrens kampledelse. Overfor BBC sier Eidevall:

– Jeg er forbauset over at Norge ikke endret noe. England spilte dem i stykker og treneren så handlingslammet ut, mener Eidevall.

Sjögren står fast på at han ikke så noen grunn til å ta grep før det var spilt 45 minutter.

– Det er som jeg svarte på spørsmålet ditt allerede. Vi valgte å gjøre det i pausen fordi det trenger en litt større justering, som vi helt enkelt trengte å tenke litt mer på når vi skulle gjøre det. Vi gjorde det i pausen, ja. Jeg kan ikke gi deg noe bedre svar, sier landslagssjefen til NRK.

NRK møtte assistenttrener Anders Jacobson i pausen.

– Hvorfor ble det ikke tatt noen grep til pause?

– Vi har gjort flere grep, vi kan bare konstatere … sa Jacobsen før han avbrøt seg selv.

– Vi må begynne med å beklage. De grepene vi har gjort, har vi ikke fått godt betalt for. Vi har gjort visse ting som vi snakket om at vi ikke skal gjøre. Der er vi nå veldig, veldig selvkritiske til at vi ikke har gjort det, helt enkelt. Nå skal vi prøve å hjelpe på en annen måte. Vi kommer til å bytte til fem bak, for å få bedre kontroll på indrekorridorene som vi vet England søker, sa Jacobsen til NRK.

Historiens største tap
Endringene, kombinert med et seierssikkert England, gjorde at det «bare» ble to baklengsmål i 2.-omgang – men det kunne blitt flere.

EM-kampen skrev seg likevel inn med blokkbokstaver – med negativt fortegn – i historiebøkene for det norske kvinnelandslaget. Norges største tap i EM var 0-4 for Tyskland fra 2009, men Norges største tap i et mesterskap var 0-5 for Kina i VM i 1999.

Det største norske tapet noensinne var 7-0 for Nederland i fjor.

Alt dette er historie etter 8-0-tap for England i EM.

Landslagssjef Martin Sjögren er langt nede etter det tunge nederlaget.

– Det er vondt å stå der ute og kjenne seg hjelpeløs. Fremfor alt så lider jeg med spillerne. Vi hadde høyere ambisjoner med denne kampen. Vi starter godt, men så går vi fra konseptene og spiller ballen bakover. Vi byr dem til å sette et ekstremt hardt trykk på oss. Vi møter et bra England, men vi gjør dem bedre enn det vi hadde trengt, sier Sjögren til NRK etter kampen.

– Dette er noe av det pinligste jeg har sett av et norsk landslag. Totalt uten eget offensivt spill. Like totalt uten defensiv struktur- og i tillegg ser veldig mange av spillerne livredde ut. Kan umulig bli verre, skriver fotballekspert Lars Tjærnås på Twitter etter Norges marerittomgang.

Landslagskaptein Maren Mjelde er svært skuffet etter kampen.

– Man føler seg ganske liten akkurat nå. Vi blir veldig hardt straffet, men det er ikke godt nok, det må vi bare erkjenne, sier hun til NRK.

– Det ble seks mål i førsteomgang, hva er det som skjer bak der?

– Jeg føler vi åpner kampen bra, og har bra trøkk, men blir veldig hardt straffet. Så får vi straffesparket imot oss, og da føler jeg at vi faller sammen. Jeg føler vi gir det til dem i avgjørende situasjoner, og så er vi ikke gode nok i egen boks. Nei, det er rett og slett ikke godt nok, sier Mjelde.

Ada Hegerberg var også langt nede etter 0-8 mot England.

– Dette er veldig ferskt, og jeg har ikke alle svar. Vi startet egentlig ganske friskt og satte dem under press, men så kom det første målet, og etter det begynte det å renne inn. Det klarte vi ikke å få stopp på, sier hun, ifølge NTB.

Sky Sports-reporter Charlotte Marsh fulgte kampen fra sidelinjen. Hun er lamslått av det hun fikk se.

– Jeg kan ikke helt tro det jeg nettopp har vært vitne til. Norge har en tropp full av spillere fra de beste lagene i verden. De har vunnet Mesterligaen flere ganger, flere ligatitler fra de beste ligaene i Europa.

Overfor NRK innrømmer Wiegman at seieren var lettere enn de forutså. Hun mener grepene til Martin Sjögren ga liten effekt.

– De endret til fem bak, men vi fikk så mye rom at vi fortsatt kunne spille vårt spill, sier Wiegman, som var overrasket over at Norge ikke presset England mer.

Ellen White, som scoret to mål mot Norge, vil ikke være med på at kampen var lettere enn hun hadde trodd på forhånd.

Overfor NRK sier hun følgende om 8-0-seieren:

– Kampen var virkelig utfordrende. Vi kom inn superfokusert, men Norge fikk en god start på deres kampinngang. De første ti minuttene var veldig harde, men det først målet ga oss virkelig masse momentum som vi bare fortsatte å bruke. Målene bare flommet på, og jeg synes prestasjonen vår var utrolig, sier hun.

En resumen: 
Gang på gang valset England gjennom et lag som posisjonerte seg feil, droppet soneforsvar og inviterte ballfører til å gjøre som hun ville. Enkeltspillere ble isolerte mot flere engelskmenn. Norge forsvarte seg én og én.

Jo, det var også personlige feil, og spillerne må ta sin del av ansvaret. Men de personlige feilene kommer som oftest når strukturen ikke er på plass. Det er landslagssjefens ansvar.

Det absurde etter 7-0-tapet for Nederland, 3-1-tapet for Tyskland og 1-0-tapet for Sverige i fjor, er at Sjögren prøvde å legge om til en fembackslinje fra og med neste kamp mot Armenia – og i de sju påfølgende kampene. Kanskje det ville gjøre at Norge stod sterkere mot bedre nasjoner?

Svaret var det motsatte. Fembackslinja ble kastet over bord. Likevel endret han til nettopp dette da det til slutt ble tatt grep mot England.

Det virker så lite bestemt, så tilfeldig. Hvis det er en plan, er den usynlig.

Hvis du lar alle 11 få forutsetninger til å vise seg fra sin beste side, blir laget åpenbart bedre. Så hvor langt er egentlig Sjögrens lag kommet?

4-0-seieren hjemme mot Belgia i fjor var et godt resultat mot en motstander i fremmarsj, men heller ikke der var Norge så imponerende som resultatet tilsier. Belgia hadde overraskende nok mest ballbesittelse, og Norge hadde 1,86 i forventede mål kontra Belgias 1,31.

2-1-seieren hjemme mot Nederland er uten tvil det sterkeste resultatet. Presset var stort. Seieren sikret VM-spill.

Men det var likevel en kamp Norge alt i alt var heldige med. Etter to raske mål, en corner og en soloprestasjon fra Isabell Herlovsen, handlet alt om Nederland. Norge var underlegne, og hvis kampen ble spilt mange ganger, hadde et fåtall endt 2-1.

For å vurdere prosjekt Sjögren må du ta noen steg tilbake og se det som en helhet. Du kan ikke bare studere England-kampen eller en eventuell brakseier mot Østerrike.

Du må se på hvilken retning landslaget har tatt, og hvilken utvikling det har hatt. Svaret er ganske enkelt: Ingen. Etter et godt 2019 har laget i beste fall stått stille samtidig som stadig flere av spillerne tar større plass i Europa.

De store nasjonene er blitt enda bedre. Nasjonene bak nærmer seg betraktelig. Norge henger ikke med.

Flere spillere har beviselig tatt store steg i klubbhverdagen. Vi har mange som presterer på et høyt, høyt nivå internasjonalt. Likevel fortsetter Sjögren å bruke flere av disse spillerne ute av sine vante posisjoner.

Como referencia: 
Los deportes de equipo, especialmente el fútbol, han sido analizados 
repetidamente desde el punto de vista de las redes complejas, buscando 
correlación entre meso y macroestructuras reticulares y el desempeño del 
equipo. Sin embargo, no se ha hecho ningún análisis con redes causales, 
buscando relaciones causa-efecto en las redes de pases y, una vez más, 
su influencia en el desempeño del equipo o su capacidad para encontrar 
descriptores macro del equipo a lo largo de una competición determinada, 
dependiendo o no del rival que haya enfrente.

Mencionar aquí un poco de la metodología.

\textbf{mi cliente como analista deportivo ---- .}


\section{Objetivos del trabajo} \label{sect:goals}

Dado el problema anterior, los objetivos a resolver son:

\begin{enumerate}
    \item \label{obj:1} Crear una herramienta para que personas del equipo técnico de
    un equipo de fútbol, como analistas de datos, analistas técnicos, analistas de rendimiento 
    físico o entrenadores, puedan tomar decisiones en base al estudio que se haga antes, durante 
    o después de un partido. 
    \item \label{obj:2} Aplicar redes bayesianas al análisis de partidos de fútbol.
\end{enumerate}

\section{Historia de la analítica en el fútbol}
Where do soccer analytics come from?
The story of soccer analytics usually starts with a cautionary tale. At 3:50 
p.m. on March 18, 1950—his notes were very precise—a retired Royal Air Force 
officer named Charles Reep, who had trained as an accountant, began 
systematically recording events at a Swindon Town match. Reep didn’t just 
want a detailed record of what happened. Like any good analyst, he wanted 
to know why it happened and what teams could learn from the data to 
improve their play. His finding over his decades of logging matches 
was that most goals resulted from possessions of three passes or fewer, 
which he took to mean that teams should simplify their tactics to get 
to goal faster, with less of the namby-pamby possession stuff. In an 
era when Hungary’s “pattern-weaving” passing style was the toast of 
the world game, Reep was writing articles with headlines like This 
Pattern-Weaving Talk is All Bunk!

The problem wasn’t Reep’s diligent data collection—it was his analysis. Most 
possessions are short, especially if you define them strictly enough to 
ensure that result; if there are a lot more short sequences, it’s no 
surprise that more goals result from them. A less polemical analyst 
might have inquired about the rate of goalscoring on possessions of various 
lengths, or how exactly those three-pass possessions developed. Reep’s own 
work showed that “60 per cent of all goalscoring moves begin 35 yards from 
an opponent’s goal,” which might make you wonder whether pattern-weaving was 
perhaps a good way to get close to goal in the first place. Not Reep, who 
worked with the great Wolves manager Stan Cullis to implement a style based 
on the principles of “direct passing.” Wolverhampton’s 
success in the fifties, including a dramatic upset of the Hungarian 
champions Honved, was seen as proof of concept and an affirmation of the 
long ball game. By his own account, Reep gathered data to “provide a 
counter to reliance upon memory, tradition and personal impressions 
that led to speculation and soccer ideologies.” But the result was 
more ideology, now with the false certainty of science.

But Reep wasn’t the first pioneer of 
soccer analytics. His own interest had been piqued by an account of Herbert 
Chapman’s statistical approach to coaching Arsenal in the early thirties. 
Just this year, the internet turned up some stunningly modern-looking 
data vizzes drawn by hand for individual matches in 1920s Hungary, Reep’s 
bête noire. A Budapest grad student named Attila Bátorfy discovered that 
dozens of the charts had run on the front page of a daily newspaper dating 
back to 1922, and were popular enough to be imitated by sports dailies in 
Italy and Sweden.

In 1996, Opta began collecting match data for the English Premier League. 
This included all of the stats that a casual soccer fan is used to – number 
of passes, number of tackles, distance travelled, etc. In many ways Opta’s 
collection of this data was the starting point of what we now refer to as 
soccer analytics.

In 2003 Michael Lewis released Moneyball and its influence quickly spread 
across the Atlantic. It is a book about how the Oakland A’s front office turned a 
low-budget baseball team into a powerhouse by using stats to recruit 
players with undervalued skillsets. The story hooked not only fans 
but also business travelers killing time in the airport bookshop 
during an era when an explosion of data had everyone scrambling to 
extract valuable analytical insights from it. The book’s release naturally started raising questions 
about the use of data in soccer. A couple of transfer market irrationalities 
were quickly exposed, most notably the salary/career length ratio for keepers 
and strikers and a strange bias towards recruiting taller players.

Fuelled by stats like Opta’s and the ideas of efficiency from baseball, 
analytics work from roughly 2003 until now has exposed many limitations 
in the popular stats. At the team level, a surprisingly low correlation 
was found between win percentage and metrics like the number of shots, 
possession, or number of passes. To this day, shot totals are used as 
some kind of proof that Team A played better than Team B. This however 
does little to explain the efficient tactics of clubs like Atletico Madrid, 
which take pride in limiting their opponents to a barrage of hopeful shots 
from outside the box while Atletico themselves focus on creating a handful 
of very good chances in the danger zone. At the player level, stats such as 
the number of tackles or distance travelled also didn’t seem to be very 
indicative of player quality. Nonetheless most of these are still used 
today in both TV broadcasts and journalistic work.


\section{Football Analytics in a nutshell}

Contextual data: main events like goals, replacement, yellow cards, etc… 
It’s characterized by the minute of the event, so quite basic. Still useful 
for news and betting companies for live broadcasts on the web.

Event data: much detailed, they track every ball action: passes, tackles, 
crosses, dribbles, shoots, etc… They are characterized by coordinates data 
(sometimes in three dimensions), so very useful for any person looking to 
analyze and create models of the game. Used by football clubs, betting 
companies, and sometimes newsrooms. Gathered by hand and algorithms.

Tracking data: every player's position and ball position at really low 
framerates sometimes 15 frames by seconds. Less mature but used by big 
clubs and researchers. Gathered either by physical sensors or software 
algorithms.

It’s often too complex and too expensive for clubs, betting companies, 
and newsrooms to gather data by themself. So a landscape of providers 
emerged too, with diverse offers, diverse data :

Stats Perform (ex Opta): the big one. It’s one of the first which started 
to gather football data at scale. They have a range of clients from top 
football clubs to big newsrooms. They are well known for both contextual 
and event data.

Statsbomb: the serious challenger. They provide detailed event data feeds 
and great analytics resources.

Skillcorner/Second Spectrum/Metrica: focused on tracking data and analysis. 
They use state-of-the-art algorithms or sensors to track all the data they 
can. The utilization of these data is infinite but complex.


The best reason to try to measure the sport is the same reason people 
used to say it couldn’t be measured: soccer is hard. Even coaches and 
analysts and scouts who’ve spent their lives learning to watch it won’t 
see games quite the same way. There are too many moving parts, too many 
possibilities to hold in your head at once. Had we but world enough and 
time, you might rewatch each match over and over to pause and study it 
and it’d still be impossible to see and remember it all. And if you have 
to do that for an opponent’s entire season, or a continent you’re charged 
with scouting? Pretty soon it starts to look like the nerds might be onto 
something.

Data scales. It doesn’t sleep or steal your food out of the office fridge. 
It’s objective-ish and consistent-ish, depending on what exactly it measures 
and how, and honestly sometimes even sketchy data is better than the stories 
we reduce soccer to without it. The hard part of analytics comes when 
we have to boil down the data to stories, too, to make it useful. How do 
we know what it means—and what we’re missing?

These days soccer data collection is big business. Companies like Opta and 
Statsbomb use human coders and computer vision to log information about 
every event that happens on the ball: passes, dribbles, shots, headers, 
fouls, tackles, interceptions, blocks, clearances, claims, and saves. 
Analysts can count these raw events, divvy them up or plot them out, 
derive second-order metrics like possession percentages, or use them 
to build sophisticated models. There’s a growing emphasis on linking 
data to video, which is undergoing a revolution of its own thanks to 
global providers like Wyscout. The juiciest stuff on the market is 
tracking data, which uses cameras or other tech to trace every 
player’s movements so that analysts can see off-ball patterns 
and soccer’s most prized commodity, space. You can join tracking 
data to event data for a more complete picture of what’s happening, 
but even that won’t tell you which way players are looking or how 
their bodies are arranged, so some researchers go even further and 
collect gaze and pose data. Because tracking data is expensive to 
collect and impossible to come by for, say, youth prospects in the 
Bolivian second division, analysts sometimes try to extract insights 
from it and apply them back to event data, which you can buy for just 
about any league with a TV contract or even a decent camera in the 
stands following the ball. But even when you’ve got information on 
what happened in a game, Reep’s story serves as a reminder that 
the challenge is what you do with it.


Este proyecto es software libre, y está liberado con la licencia \cite{gplv3}.
