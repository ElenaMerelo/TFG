\chapter{Introducción}

En este capítulo se describen la motivación del proyecto y los objetivos que
se quieren alcanzar durante el desarrollo, así como una introducción histórica al
a la aplicación del análisis al fútbol y las redes bayesianas. 

\section{Motivación}
%¿Por qué queréis hacer este TFG? ¿A quién ayuda? ¿Quién lo usaría? ¿Qué solución proponéis? La ingeniería del software trata de resolver problemas, no de hacer aplicaciones, y los problemas deben estar antes que nada.

El fútbol es un deporte de equipo  \textit{complicado}. Suponiendo que jueguen veintidós personas, como parte del equipo técnico habrá que decidir quiénes jugarán, dónde se posicionará cada uno, cuándo y qué cambios se harán, si se aboga más por una táctica defensiva u ofensiva, así como en qué hacer incidencia en los entrenamientos, entre otros. Igualmente, más a nivel de jugador, habrá que medir fuerzas, ir decidiendo a quién y cuándo pasar, desde dónde y en qué momento tirar a portería, en qué momento atacar, si regatear o no, etc. Son bastantes las variables que entran, literalmente, en juego. 

Surgen pues de manera natural preguntas en torno a la práctica de este deporte alrededor del cual gira todo un mundo, y la importancia de las decisiones que se van tomando tanto dentro como fuera del campo. Principalmente,  el presente trabajo pretende ayudar a un entrenador que , durante un partido, quiera saber el efecto de un cambio de alineación, técnica o formación, y tras el mismo saber qué mejorar, en qué trabajar más, qué ha causado un gol o pérdida de la posesión, o a un analista de datos que, antes de un partido, estudia al equipo rival o al propio, a nivel individual o colectivo: el número de contactos de cada jugador con el balón, el porcentaje de acierto tanto tirando a gol como pasando a un compañero, la dirección, el número de ataques realizados por sector del campo, las conexiones entre jugadores,...

No obstante, podría ser usado también por aficionados, personas que apuestan, analistas tácticos, técnicos o de rendimiento físico, y no necesariamente de un equipo profesional que tenga muchos medios. Pero de esto hablaremos en profundidad en próximos capítulos, cuando establezcamos los objetivos, clientes e historias de usuario.
% no me convence el final, me falta responder las últimas preguntas

\section{Objetivos del trabajo} \label{sect:goals}

Dado el problema anterior, los objetivos a resolver son:

\begin{enumerate}
    \item \label{obj:1} Crear una herramienta para que personas del equipo técnico de
    un equipo de fútbol, como analistas de datos, analistas técnicos, analistas de rendimiento 
    físico o entrenadores, puedan tomar decisiones en base al estudio que se haga antes, durante 
    o después de un partido. 
    \item \label{obj:2} Entender cómo se pueden aplicar redes causales e inferencia 
    bayesiana en deportes de equipo, y adaptar resultados en ese campo a 
    los tipos de datos que existen aquí.
\end{enumerate}

\section{Historia de la analítica en el fútbol}


Este proyecto es software libre, y está liberado con la licencia \cite{gplv3}.
