\chapter{Introducción}

\section{Motivación}
El fútbol es un deporte de equipo  \textit{complicado}. Suponiendo que 
jueguen veintidós personas, como parte del equipo técnico habrá que 
decidir quiénes jugarán, dónde se posicionará cada uno, cuándo y qué 
cambios se harán, si se aboga más por una táctica defensiva u ofensiva, 
así como en qué hacer incidencia durante los entrenamientos, entre otros. 
Son bastantes las variables que entran, literalmente, en juego. 

Surgen, pues, de manera natural preguntas en torno a la práctica de 
este deporte alrededor del cual gira todo un mundo, y la importancia 
de las decisiones que se van tomando tanto dentro como fuera del 
campo: Si a un equipo le va mal en un campeonato, se echa al entrenador, pero ¿de quién 
es realmente 'la culpa'? ¿Qué cambios se pueden hacer? ¿Qué táctica es mejor? ¿Cómo se influencian los 
jugadores entre ellos, o cuando juegan con otro equipo? ¿Es ello determinante en los 
resultados que consiguen a lo largo de una temporada? 
Principalmente,  el presente trabajo pretende ayudar a un 
entrenador que, durante un partido, quiera saber el efecto de una 
alineación, técnica o formación, y tras el mismo saber 
qué mejorar, en qué trabajar más, qué ha causado un gol o pérdida 
de la posesión, o a un analista de datos que, antes de un partido, 
estudia al equipo rival o al propio, a nivel individual o colectivo: 
el número de contactos de cada jugador con el balón, el porcentaje 
de acierto tanto tirando a gol como pasando a un compañero, la 
dirección, el número de ataques realizados por sector del campo, 
las conexiones entre jugadores y así. Todo ello con énfasis en las 
redes de pases, viendo qué información podemos obtener de ellas y su 
importancia.

\section{Objetivos del trabajo} \label{sect:goals}

Dado el problema anterior, los objetivos que nos planteamos son:

\begin{enumerate}
    \item \label{obj:1} Crear una herramienta para que personas del equipo 
    técnico de un equipo de fútbol, como analistas de datos, analistas técnicos, analistas de rendimiento 
    físico o entrenadores, puedan tomar decisiones en base al estudio que se haga antes, durante 
    o después de un partido. 
    \item \label{obj:2} Entender cómo se pueden aplicar técnicas estadísticas en deportes de equipo, 
    y adaptar resultados en ese campo a los tipos de datos que existen aquí.
\end{enumerate}

Este proyecto es software libre, y está liberado con la licencia \cite{gplv3}.
