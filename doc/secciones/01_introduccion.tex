\chapter{Introducción}
Me gusta jugar al fútbol, estudio informática y matemáticas, con lo que 
surgen de manera natural preguntas en torno a la práctica de este deporte 
alrededor del cual gira todo un mundo, y la importancia de las decisiones 
que se van tomando tanto dentro como fuera del campo.

El análisis del fútbol no es para nada algo nuevo, y no es otra cosa que 
pensar, observar y reflexionar sobre lo que ocurre a lo largo de partidos
o temporadas, para actuar consecuentemente y, por ejemplo, fichar a 
ciertos jugadores en detrimento de otros, centrarse más en la defensa 
que en el ataque, entrenar un tipo de jugada o pases específicos, darle 
importancia a la táctica o a la condición física. La lista puede ser muy 
larga, tanto como los factores que, literalmente, entran en juego.

Así pues, este proyecto tiene como idea ayudar desde los jugadores hasta 
el cuerpo técnico del equipo: entrenadores, director técnico, preparadores 
físicos, analistas tácticos y de datos, scouters. Pasamos a detallar los roles
principales y que pueden necesitar más explicación.
\begin{itemize}
    \item Analista táctico: se encarga de estudiar los equipos y cómo se 
    desenvuelven en los diferentes partidos.
    \item Scouter: es la persona encargada de la búsqueda y captación de 
    jugadores. Su tarea consiste en encontrar los jugadores que necesita 
    el equipo o el club, por lo que analiza sus cualidades y posibilidades 
    de integración al equipo, desde su rendimiento futbolístico hasta el 
    económico.
    \item Analista de datos: toma la información estadística de cada 
    partido tanto en lo individual como en lo colectivo. Establece 
    relaciones para encontrar respuestas o ideas que puedan colaborar con 
    la toma de decisiones del club referentes a lo táctico, lo técnico y 
    lo económico, para el modelo de juego del equipo o para la compra y 
    venta de jugadores.
\end{itemize}

Manteniendo esto en mente, nuestra solución podría ser usada por ellos. 
Vemos claramente que el trabajo se puede enfocar de diversas maneras, 
y conforme avancemos habremos de descartar algunas y quedarnos con otras, 
dejando todo debidamente justificado. Dependerá de los datos que se 
encuentren disponibles, y la necesidad que haya en el mercado. 
Normalmente y si consultamos la literatura, los análisis son estáticos; 
rendimiento y mapas de calor de un jugador, desde dónde se ha lanzado más 
a portería, etc. pero no hay tanto estudiado en cuanto a causalidad; es por 
esto que surgió la idea de hacer el presente trabajo. 

Idealmente, haríamos una aplicación móvil de manera que el análisis esté 
fácil y rápidamente accesible, pero no nos va a dar tiempo, con lo que en 
un principio desarrollaremos una librería.
