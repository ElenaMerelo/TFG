\chapter{Análisis del problema}
 
\subsection{Redes bayesianas}
Las redes bayesianas nos ayudan a modelar y entender las muchas variables que informan nuestro proceso de 
toma de decisiones. Las decisiones más complejas están normalmente basadas en una multitud de factores o 
variables. Por ejemplo, imaginemos que el dueño de un equipo de fútbol tiene que decidir antes del comienzo 
de cada temporada cuánto dinero hay que invertir en nuevos jugadores. Habrá de considerar los ingresos 
probables por las ventas de jugadores no deseados; el gasto neto relativo de otros
equipos; y el posible impacto negativo en el desempeño del equipo de hacer demasiados cambios de personal de una 
vez, entre otros. Podemos mapear la decisión que tiene que hacer, y todas las diferentes variables, usando 
una red bayesiana, esto es, un modelo gráfico que captura la relación entre variables que están bajo 
supuestos de causalidad o influencia.\cite{things-to-know-BN}

Básicamente entonces, \textbf{una red bayesiana es un diagrama que 
usa flechas o arcos dirigidos para mostrar cómo distintos factores, representados por nodos elípticos, se 
influencian los unos a los otros.} Cada nodo viene con su tabla de probabilidad, la cual refleja las 
posibilidades de varios desenlaces, provenientes de las influencias que le afectan directamente. Una vez 
la estructura del grafo y la tabla de una red han sido definidos, hay algoritmos estándar que 
calculan los estados de las variables desconocidas basándose en los estados de las variables conocidas en el
modelo.\cite{learning-algorithms-BN-comparison}, \cite{BN-achilles-heel}, \cite{different-algorithmic-schemes}

Una de las razones por las que las redes bayesianas son tan potentes es que pueden realizar inferencias 
tanto predictivas como diagnósticas. Por ejemplo, podemos por un lado predecir la posición en la liga de un equipo para 
un valor dado (observación) de rendimiento, y por otro ingresar un estado de posición en la 
liga como observación para examinar qué nivel de desempeño del equipo podría explicarla. Estos algoritmos estándar son
llamados algoritmos de "propagación bayesiana"\cite{Cano2004},\cite{more-algorithms}, \cite{back-prop} porque se basan en el teorema de Bayes, en el que la 
probabilidad de una variable desconocida se actualiza después de que se obtenga evidencia relevante para esa variable.\cite{prop-alg}
