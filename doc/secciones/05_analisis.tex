\chapter{Desarrollo teórico}
En este capítulo estableceremos la base matemática del proyecto. Principalmente nos basaremos en 
los libros \textit{Advanced Data Analysis
from an Elementary Point of View} \cite{ada} y \textit{Probabilistic Networks for Practitioners — A
Guide to Construction and Analysis of Bayesian
Networks and Influence Diagrams} \cite{pgm}, junto con el capítulo \textit{An overview of the representation and 
discovery of causal relationships using bayesian networks} \cite{cooper}.

\subsubsection{Variables}
Una variable representa un conjunto exhaustivo de eventos mutuamente excluyentes, conocidos como dominio 
de la variable. Estos eventos también se pueden llamar estados, niveles, valores, elecciones o opciones. 
El dominio de una variable puede ser discreto o contínuo; los dominios discretos son siempre finitos. 
Usaremos letras mayúsculas para denotar variables o conjuntos de variables, y letras minúsculas para sus 
valores. Consecuentemente, $X = x$ puede denotar el hecho de que la variable $X$ toma el valor $x$ o que 
el conjunto de variables $X = (X_{1},...,X_{n})$ toma el vector de valores $x = (x_{1},...,x_{n})$. 
Con $dom(X)= (x_{1},...,x_{||X||})$ denotaremos el dominio de $X$, donde $||X|| = |dom(X)|$ es el número 
de posibles valores de $X$. Si $X = (X_{1},...,X_{n})$, entonces $dom(X)$ es el producto cartesiano sobre 
los dominios de las variables en $X$, $$dom(X)=dom(X_{1})\times ... \times dom(X_{n})$$, y por lo tanto 
$||X|| = \prod_i ||X_{i}||$. Para dos (conjuntos de) variables $X$ e $Y$ escribiremos $dom(X \cup Y)$ 
o $dom(X,Y)$ para denotar $dom(X) \times dom(Y)$.

Usaremos la noción de variable y vértice o nodo de manera indistinta.  

\section{Fundamentos de probabilidad} 
El hecho de que la estructura de una red probabilística pueda ser caracterizada como un DAG proviene de axiomas básicos 
de cálculo de probabilidades que conduce a la factorización recursiva de una distribución de probabilidad conjunta en 
un producto de distribuciones de probabilidad condicionada de menor dimensión. En primer lugar, cada distribución de 
probabilidad conjunta se puede descomponer o factorizar en un producto de distribuciones condicionadas de distinta 
dimensión, donde la dimensión de la distribución más grande es idéntica a la dimensión de la distribución conjunta. En 
segundo lugar, las declaraciones de independencia condicionada locales se manifiestan como reducciones en las dimensiones 
de algunas de las distribuciones de probabilidad condicionada. Estos dos hechos dan lugar a la estructura de un DAG.

De hecho , una distribución de probabilidad conjunta $P$ se puede descomponer recursivamente de esa manera sí y solo sí 
hay un DAG que represente correctamente la dependencia (condicionada) y las propiedades de independencia de $P$. Esto 
significa que un conjunto de distribuciones de probabilidad condicionada con respecto a un DAG $G=(V,E)$, (esto es, 
una distribución $P(A|pa(A))$ para cada $A \in V$) define una distribución de probabilidad conjunta.

Consecuentemente, un modelo de redes probabilísticas puede ser especificado directamente a través de la distribución 
de probabilidad conjunta, o mediante un DAG que muestre las relaciones causa-efecto junto con un conjunto de 
las correspondientes distribuciones de probabilidad condicionada.

Pasamos a presentar algunos axiomas de cálculo de probabilidades que dan lugar al teorema de Bayes, así como 
la regla para la descomposición de una distribución de probabilidad conjunta en productos de distribuciones 
condicionadas. Adicionalmente, veremos las operaciones fundamentales necesitadas para realizar inferencia 
sobre redes probabilísticas.

\subsection{Axiomas} 
Dado un evento $b$, la probabilidad condicionada de una evento $a$ es $x$, $P(a|b) = x$.
Los siguientes tres axiomas dan la base para el cálculo de probabilidades bayesianas.
\begin{axioma} 
Para cualquier evento $a$, $0 \leq P(a) \leq 1$, con $P(a)=1 \leftrightarrow a$ ocurre con certeza.
\end{axioma}

\begin{axioma}
Para dos eventos $a$ y $b$ mutuamente excluyentes, la probabilidad de que $a$ o $b$ ocurran es 
$$P(a or b) \equiv P(a \vee b)= P(a) + P(b)$$.
En general, si los eventos $a_{1},...,a_{n}$ son incompatibles dos a dos, entonces 
$$P(\bigcup_{i}^{n} a_{i})= P(a_{1}) + ... + P(a_{n}) = \sum_{i}^{n} P(a_{i})$$.
\end{axioma}

\begin{axioma}
Para cualesquiera dos eventos $a$ y $b$, la probabilidad de que $a$ y $b$ ocurran es 
$$P(a and b) \equiv P(a,b) = P(b|a)P(a) = P(a|b)P(b)$$, 
donde $P(a,b)$ se llama probabilidad conjunta de los eventos $a$ and $b$.
\end{axioma}

\subsection{Distribuciones de probabilidad para variables}



\section{Fundamentos de redes probabilísticas}
%¿Por qué nos interesa esto? ¿Qué vamos a hacer con ellas?
%¿Qué tienen que ver con los grafos? O igual simplemente haz una referencia.
Las redes probabilísticas son modelos gráficos de interacciones (causales) entre una
conjunto de variables, donde las variables se representan como vértices o nodos de un 
gráfico y las interacciones (dependencias directas) como aristas dirigidas (también
enlaces y arcos) entre los vértices. Cualquier par de vértices no conectados indican 
independencia (condicional) entre las variables representadas
por estos vértices en circunstancias particulares que se pueden leer fácilmente desde el
grafo. 

Los grafos han demostrado ser un lenguaje muy intuitivo para representar
tales declaraciones de dependencia e independencia, y por lo tanto proporcionan un excelente
lenguaje para comunicar y discutir las relaciones entre las 
variables del dominio del problema, las cuales se pueden representar de forma muy compacta 
mediante grafos dirigidos acíclicos (DAG).

No obstante, la propiedad que distingue a la inferencia en redes probabilísticas
de otros paradigmas de razonamiento automático es su capacidad para hacer razonamientos intercausales: 
obtener evidencia que apoye una sola hipótesis o un subconjunto de
hipótesis conduce a la disminución de la creencia en las otras con las que compiten y para las que no se 
han encontrado unas bases que las soporten. Por ejemplo, hay un gran número de posibles causas por las 
que un coche puede no arrancar, entre ellas falta de combustible. Observar que el indicador de combustible 
indica que no hay combustible proporciona una fuerte evidencia de que la falta de combustible es la causa 
del problema, a la vez que la creencia en otras causas posibles disminuye sustancialmente. La capacidad 
de las redes probabilísticas para realizar automáticamente tal inferencia intercausal de manera adecuada 
es una contribución clave a su poder de razonamiento.

La parte gráfica de una red probabilística se conoce como aspecto cualitativo, mientras que la 
parte probabilística o numérica constituye su aspecto cuantitativo. Nos dedicaremos primeramente al aspecto 
cualitativo de las redes probabilísticas.

\section{Redes causales}

\section{Fundamentos de redes bayesianas}
Las redes bayesianas nos ayudan a modelar y entender las muchas variables que informan nuestro proceso de 
toma de decisiones. Las decisiones más complejas están normalmente basadas en una multitud de factores o 
variables. Por ejemplo, para el presidente de un equipo de fútbol \ref{hu:presidente}, podemos 
mapear la decisión que tiene que tomar y las diferentes variables usando 
una red bayesiana, esto es, un modelo gráfico que captura la relación entre variables que están bajo 
supuestos de causalidad o influencia \cite{things-to-know-BN}.

Básicamente entonces, \textbf{una red bayesiana es un diagrama que 
usa flechas o arcos dirigidos para mostrar cómo distintos factores, representados por nodos elípticos, se 
influencian los unos a los otros.} Cada nodo viene con una tabla de probabilidades condicionadas, la cual refleja las 
posibilidades de varios desenlaces, provenientes de las influencias que le afectan directamente. Una vez 
la estructura del grafo y dicha tabla han sido definidas, hay algoritmos estándar que 
calculan los estados de las variables desconocidas basándose en los estados de las variables conocidas en el
modelo \cite{learning-algorithms-BN-comparison}, \cite{BN-achilles-heel}, \cite{different-algorithmic-schemes}.

Una de las razones por las que las redes bayesianas son tan potentes es que pueden realizar inferencias 
tanto predictivas como diagnósticas. Por ejemplo, podemos por un lado predecir la posición en la liga de un equipo para 
un valor dado (observación) de rendimiento, y por otro ingresar un estado de posición en la 
liga como observación para examinar qué nivel de desempeño del equipo podría explicarla. Estos algoritmos estándar son
llamados algoritmos de ``propagación bayesiana" \cite{Cano2004}, \cite{more-algorithms}, \cite{back-prop} porque se basan en el teorema de Bayes, en el que la 
probabilidad de una variable desconocida se actualiza después de que se obtenga evidencia relevante para esa variable \cite{prop-alg}.

Las clases de causalidad que producen redes bayesianas son:
\begin{itemize}
    \item \textbf{Cadena causal}: describe variables que tienen un efecto dominó las unas sobre las otras. Por ejemplo, \textit{cambios en la calidad de 
    los jugadores} tiene impacto sobre el \textit{desempeño del equipo} que a su vez influencia la \textit{posición en la liga}. 
    Esto quiere decir que la \textit{posición en la liga} es independiente de los \textit{cambios en la calidad de los jugadores} una vez conocemos el \textit{desempeño del equipo}.\\
    cambios en la calidad de los jugadores $\rightarrow$ Desempeño del equipo $\rightarrow$ Posición en la liga
    \item \textbf{Efecto común}: ocurre cuando dos variables diferentes, tales como \textit{fichajes} y \textit{jugadores vendidos}, tienen influencia sobre una tercera variable tal como 
    \textit{gasto neto en transferencias}. Esto significa que \textit{jugadores vendidos} depende de \textit{fichajes} una vez que conocemos el \textit{gasto neto en transferencias}.\\
    Fichajes $\rightarrow$ Gasto neto en transferencias $\leftarrow$ Jugadores vendidos    
    \item \textbf{Causa común}: tiene lugar cuando dos variables distintas, tales como \textit{posición en la liga} y \textit{asistencia}, se ven influenciados por la misma variable, tal 
    como \textit{desempeño del equipo}. Ello significa que \textit{asistencia} es independiente de \textit{posición en la liga} una vez conocemos \textit{desempeño del equipo}.\\
    Posición en la liga $\leftarrow$ Desempeño del equipo $\rightarrow$ Asistencia
\end{itemize}


\subsection{Construcción de redes bayesianas}
Construir una red bayesiana implica determinar su estructura y su tabla de probabilidades condicionadas. Podemos hacer esto 

\section{Resolución de redes probabilísticas}

