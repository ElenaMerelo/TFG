\chapter{Desarrollo teórico}
En este capítulo estableceremos la base matemática del proyecto. Principalmente nos basaremos en 
los libros \textit{Análisis de datos avanzado desde un punto de vista elemental} \cite{ada}, 
\textit{Learning Bayesian Networks} \cite{neapolitan} 
y \textit{Redes probabilísticas para aficionados - Una guía para la construcción y análisis de redes 
bayesianas y diagramas de influencia} \cite{pgm}, junto con el capítulo 
\textit{Visión general de la representación y descubrimiento de relaciones causales usando redes 
bayesianas} \cite{Cooper} y la tesis \textit{Modelos de clasificación de documentos basados en redes 
bayesianas} \cite{tesis-alfonso}.

\section{Teoría de la probabilidad}

\subsection{Conceptos básicos}
En esta subsección revisaremos algunos conceptos de teoría de la probabilidad básicos, en el 
caso discreto. 

\subsubsection{Funciones y espacios de probabilidad}
Un {\em espacio muestral} es un conjunto contable\footnote{Un conjunto es {\em contable} si tiene el mismo número de 
elementos (cardinalidad) que algún subconjunto de los números naturales, $\mathbb{N}$.} $\Omega = \{x_1, x_2, 
\dots \}$ que representa los posibles resultados de un experimento.

Una función $P : 2^\Omega \longrightarrow \mathbb{R}$, donde $2^\Omega$ es el conjunto de todos los 
subconjuntos (eventos) de $\Omega$, se denomina {\em función de probabilidad} si satisface los axiomas de 
probabilidad de Kolmogorov:

\begin{enumerate}
    \item $0 \leq P(A) \leq 1, \forall A  \subseteq \Omega$.
    \item $P(\Omega) = 1$.
    \item Si $E_1, E_2, \dots$ son tales que $E_i \cap E_j = \emptyset, \quad \forall i\neq j$, entonces
    $$P\left(\cup_i E_i\right) = \sum_i P(E_i).$$
\end{enumerate}

Si $P$ es una función probabilística, el par $(\Omega, P)$ se llama {\em espacio de probabilidad}.

\subsubsection{Probabilidad condicionada y teorema de Bayes}\label{condi}

Sean $A, B \subseteq \Omega$ tales que $P(B) \neq 0$. Entonces, la {\em probabilidad condicionada} de $A$, 
dado $B$ y notada como $P(A|B)$ viene dada por:

$$P(A|B) = \frac{P(A \cap B)}{P(B)}.$$

Diremos que dos eventos $A,B$, donde $P(A) \neq 0$ y $P(B) \neq 0$ son {\em independientes} si $P(A|B) = P(A)$, 
y son {\em condicionalmente independientes} si dado $C$, $P(A|B \cap C) = P(A|C)$.

Si $E_1, \dots, E_n$ es un conjunto de eventos mutuamente disjuntos tales que $\cup_i E_i = \Omega$, esto es, 
son una {\em partición} del espacio muestral, y $P(E_i) > 0, \forall i = 1, \dots, n$, entonces para cualquier 
evento $A$: 

\begin{equation}
P(A) = \sum_{i=1}^n P(A \cup E_i) = \sum_{i=1}^n P(A|E_i)P(E_i). 
\label{lawoftotalprobability}
\end{equation}

Esta propiedad se llama {\em regla de probabilidad total}, y puede ser demostrada con teoría básica de conjuntos.

\subsubsection{Variables aleatorias}

Una {\em variable aleatoria} (discreta) es una función que mapea eventos a valores de un conjunto contable, donde 
cada valor tiene una probabilidad mayor o igual a cero. Representaremos las variables aleatorias con letras 
mayúsculas, y los valores a los que los eventos son mapeados, con mínusculas. 

Una variable aleatoria induce una nueva función de probabilidad sobre sus valores. Usaremos la notación $X=x$ para 
representar el conjunto de eventos de $\Omega$ mapeados por $X$ a $x$. Si definimos $P_X(\{x\}) := P(X=x)$, 
entonces $P_X$ es una función de probabilidad. De ahora en adelante escribiremos simplemente $P(X=x)$, que se 
denominará la {\em distribución de probabilidad} de $X$ (o solo {\em distribución} de $X$). Si sabemos que $X$ 
es una variable aleatoria, podemos denotar su distribución como $P(X)$ sin ambigüedad.

En ocasiones, dada una variable aleatoria $X$ y siendo $x$ uno de sus valores, escribiremos para ser más breves 
$p(x)$ en vez de $P(X=x)$. Esta cantidad es la {\em probabilidad de $x$}.

Asimismo, para una variable aleatoria $A$, denominaremos al conjunto de valores que $A$ puede tener como 
``rango de $A$'', $\Omega_A$.

Dadas $X,Y$ variables aleatorias definidas en el mismo espacio muestral, y dos valores $x$ de $X$ e $y$ de 
$Y$, podemos medir la probabilidad del evento intersección $P(X=x, Y=y)$, la cual se denomina {\em distribución 
de probabilidad conjunta} de $X$ e $Y$. Podemos definir la distribución de probabilidad conjunta de un conjunto 
arbitrario de variables aleatorias $\{X_1, X_2, \dots, X_n \}$ como $P(X_1=x_1, X_2=x_2,\dots, X_n=x_n )$.

Pasamos a introducir la operación conocida como {\em marginalización}. Dada una distribución de probabilidad 
conjunta $P(X=x,Y=y)$, podemos obtener la distribución $P(X=x)$ (llamada {\em distribución marginal de 
probabilidad de $X$}\footnote{El término ``marginal'' se usa cuando una distribución se obtiene por marginalización 
de otra distribución conjunta, pero de hecho es también una distribución ``convencional''.}), sumando sobre todos 
los valores de $y$ en el rango de $Y$: 

$$P(X=x) = \sum_y P(X=x, Y=y).$$

Esta es otra expresión de la ley de probabilidad total \ref{lawoftotalprobability}.

En último lugar, una vez entendida la analogía entre variable aleatoria y eventos, podemos definir la 
{\em distribución de probabilidad condicionada} de una variable aleatoria $X$ dado $Y$:

$$P(X=x|Y=y) = \frac{P(X=x, Y=y)}{P(Y=y)}.$$

Este caso se puede extender a un conjunto de dos o más variables. Un resultado que permite calcular la 
distribución conjunta de un conjunto de variables aleatorias usando únicamente probabilidades condicionadas 
es la {\em regla de la cadena}:

$$P(X_1=x_1, \dots, X_n=x_n) = \prod_{i=1}^n P(X_i=x_i|X_{i-1}=x_{i-1}, \dots, X_1=x_1),$$ la cual puede 
probarse fácilmente como aplicación directa del teorema de Bayes \ref{eq:bayes}.

\subsubsection{Independencia condicionada y observaciones de variables aleatorias}

Dos de las nociones ``clásicas'' en teoría de la probabilidad (y ya mencionadas en la sección \ref{condi}) son 
el concepto de {\em independencia} y {\em independencia condicionada}. 

Desde el punto de vista de variables aleatorias, podemos reescribir la definición de {\em eventos independientes}; 
diremos que dos eventos $X=x$ e $Y=y$ (siendo $x,y$ dos valores de las variables aleatorias $X,Y$) son 
independientes si 

$$p(x,y) = p(x) \, p(y).$$

Además, $X$ e $Y$ son {\em independientes} sí y solo sí $$P(X=x, Y=y) = P(X=x) \, P(Y=y),$$ 
para todos los valores $x,y$ de $X$ e $Y$, respectivamente.

Para problemas con al menos tres variables aleatorias, $X,Y,Z$, puede observarse {\em independencia 
condicionada}; diremos que, dada $Z$, $X$ e $Y$ son {\em independientes condicionadamente } $\leftrightarrow$ todo par 
de valores $x$ (de $X$) e $y$ (de $Y$) es independiente condicionadamente para cada $z$ (de $Z$), tal que 
$p(z) > 0$, esto es:

$$\forall x, y, z, \quad p(z) > 0 \,\, \Rightarrow \,\, p(x,y|z) = p(x|z)\, p(y|z).$$

Este concepto no está solo limitado a tres variables, y se puede extender a conjuntos grandes.

Diremos que una variable es {\em observada} si toma uno de los valores de su rango. Si queremos testear la {\em 
independencia con respecto a una variable observada}, necesitamos únicamente testear la independencia con respecto 
a ese valor asignado. Es de notar que observar una variable puede cambiar las relaciones de independencia entre
un conjunto de variables.

Escribiremos $A\perp B | C$ cuando una variable $A$ es independiente de otra, $B$, tras la observación de $C$. Si 
$A$ es independiente de $B$, sin observaciones adicionales, escribiremos $A \perp B$.

Dados $\mathbf{W, X, Y, Z}$ conjuntos de variables aleatorias, las relaciones de independencia tienen 
las siguientes propiedades:

\begin{itemize}
 \item Simetría $(\mathbf{X} \perp \mathbf{Y} | \mathbf{Z}) \Rightarrow (\mathbf{Y} \perp \mathbf{X} | \mathbf{Z})$.
 \item Descomposición $(\mathbf{X} \perp \mathbf{Y, W} | \mathbf{Z}) \Rightarrow (\mathbf{X} \perp \mathbf{Y} | \mathbf{Z})$.
\end{itemize}

\section{Redes bayesianas}

\subsection{Introducción}
Las redes bayesianas nos ayudan a modelar y entender las muchas variables que informan nuestro proceso de 
toma de decisiones. Las decisiones más complejas están normalmente basadas en una multitud de factores o 
variables. Por ejemplo, para el presidente de un equipo de fútbol \ref{hu:presidente}, podemos 
mapear la decisión que tiene que tomar y las diferentes variables usando 
una red bayesiana, esto es, un modelo gráfico que captura la relación entre variables que están bajo 
supuestos de causalidad o influencia \cite{things-to-know-BN}.

Como ya hemos comentado en \ref{subsect:BN}, \textbf{una red bayesiana es un diagrama que 
usa flechas o arcos dirigidos para mostrar cómo distintos factores, representados por nodos elípticos, se 
influencian los unos a los otros.} Cada nodo viene con una tabla de probabilidades condicionadas, la cual 
refleja las posibilidades de que tenga lugar distintos desenlaces, provenientes de las influencias que 
le afectan directamente. Una vez la estructura del grafo y dicha tabla han sido definidas, hay algoritmos 
estándar que calculan los estados de las variables desconocidas basándose en los estados de las variables 
conocidas en el modelo \cite{learning-algorithms-BN-comparison}, \cite{BN-achilles-heel}, 
\cite{different-algorithmic-schemes}.

Una de las razones por las que las redes bayesianas son tan potentes es que pueden realizar inferencias 
tanto predictivas como diagnósticas. Por ejemplo, podemos por un lado predecir la posición en la liga de 
un equipo para un valor dado de rendimiento, y por otro ingresar un estado de posición en la 
liga como observación para examinar qué nivel de desempeño del equipo podría explicarla. Estos algoritmos 
estándar son llamados algoritmos de ``propagación bayesiana" \cite{Cano2004}, \cite{more-algorithms}, 
\cite{back-prop} porque se basan en el teorema de Bayes; en ellos, la probabilidad de una variable 
desconocida se actualiza después de que se obtenga evidencia relevante para esa variable \cite{prop-alg}.

Las clases de causalidad que producen redes bayesianas son:
\begin{itemize}
    \item \textbf{Cadena causal}: describe variables que tienen un efecto dominó las unas sobre las otras. 
    Por ejemplo, \textit{cambios en la calidad de los jugadores} tiene impacto sobre el \textit{desempeño 
    del equipo} que a su vez influencia la \textit{posición en la liga}. 
    Esto quiere decir que la \textit{posición en la liga} es independiente de los \textit{cambios en la 
    calidad de los jugadores} una vez conocemos el \textit{desempeño del equipo}.\\
    cambios en la calidad de los jugadores $\rightarrow$ Desempeño del equipo $\rightarrow$ Posición en la 
    liga
    Se dirá que son {\em nodos en secuencia}. \label{nodos-secuencia}
    \item \textbf{Efecto común}: ocurre cuando dos variables diferentes, tales como \textit{fichajes} y \textit{jugadores vendidos}, tienen influencia sobre una tercera variable tal como 
    \textit{gasto neto en transferencias}. Esto significa que \textit{jugadores vendidos} depende de \textit{fichajes} una vez que conocemos el \textit{gasto neto en transferencias}.\\
    Fichajes $\rightarrow$ Gasto neto en transferencias $\leftarrow$ Jugadores vendidos 
    Se dirá que son {\em nodos convergentes}. \label{nodos-conv}   
    \item \textbf{Causa común}: tiene lugar cuando dos variables distintas, tales como \textit{posición en la liga} y \textit{asistencia}, se ven influenciados por la misma variable, tal 
    como \textit{desempeño del equipo}. Ello significa que \textit{asistencia} es independiente de \textit{posición en la liga} una vez conocemos \textit{desempeño del equipo}.\\
    Posición en la liga $\leftarrow$ Desempeño del equipo $\rightarrow$ Asistencia
    Se dirá que son {\em nodos divergentes}. \label{nodos-div}
\end{itemize}

\subsection{Factorización de redes bayesianas}
Nos interesa reducir el número de parámetros que encontrarmos en las redes bayesianas, de manera que sean menos 
costosas de analizar, y la predicción sea mejor. En este subsección estudiamos maneras de hacerlo.

Dado un conjunto de variables aleatorias $\mathbf{U} = \{X_1, \dots, X_n\}$, un {\em modelo de dependencia} $M$ 
es un conjunto de relaciones de independencia suficiente para determinar, para cualesquiera $\mathbf{A_1}, 
\mathbf{A_2}, \mathbf{A_3} \subset \mathbf{U}$ con $\mathbf{A_1}, \mathbf{A_2}, \mathbf{A_3}$
disjuntos, y $\mathbf{A_1}, \mathbf{A_2}$ no vacíos, si la relación de independencia $\mathbf{A_1} \perp 
\mathbf{A_2} | \mathbf{A_3}$ es verdadera o falsa. 

Si tenemos un conjunto de variables aleatorias cada una en un nodo de un DAG \ref{def:grafo}, la {\em condición 
de Markov} nos dice que una variable aleatoria de un grafo es independiente de sus no-descendientes, dados 
sus padres \cite{markov-assumption}. 

Una distribución de probabilidad $P$ en un DAG $D$ constituye una {\em red bayesiana} $\leftrightarrow$ $D$ es un 
I-map mínimo de $P$. Esto nos dice que una red bayesiana sobre un conjunto de variables aleatorias puede verse 
como un conjunto de relaciones de independencia almacenadas usando un DAG \cite{i-map}.

\begin{teorema}\emph{(Factorización de una red bayesiana)} 
La distribución conjunta de una red bayesiana $P(X_1, \dots, X_n)$ se factoriza de la siguiente manera:

$$P(X_1, \dots, X_n) = \prod_{i=1}^n P(X_i|pa(X_i))$$

donde $pa(X_i)$ es el conjunto de padres de la variable en el grafo. 
\end{teorema}

El recíproco también es cierto:

\begin{teorema} Sea $P$ una distribución en un grafo tal que $P(X_1, \dots, X_n) = \prod_{i=1}^n P(X_i|Pa(X_i))$. Entonces, el grafo es un I-map de $P$.
\end{teorema}

Las redes bayesianas cumplen el conjunto de condiciones de Markov, si bien suele aplicarse (y son equivalentes) 
más el concepto de $d$-separación, un criterio gráfico de independencia, por su buen rendimiento computacional 
\cite{pearl88}.

\subsection{Criterios gráficos de independencia}

Las redes bayesianas nos dan un conjunto de {\em criterios gráficos} para comprobar si, dado otro conjunto, dos 
conjuntos de variables aleatorias son independientes.

Diremos que $X$ está $d$-separado (o simplemente {\em separado}) de $Y$ dados $Z_1, \dots, Z_k$ si todos los 
caminos entre $X$ e $Y$ recorriendo los arcos de la red en ambas direcciones , están {\em bloqueados} por cualquier 
nodo de $Z_1, \ldots, Z_k$. Las dos maneras en que un nodo puede bloquear un camino son: 

\begin{itemize}
\item Hay observaciones de nodos en secuencia en el camino.
\item Hay un nodo convergente sin observar, cono todos sus descendientes no observados
\end{itemize}

Estamos pues en condiciones de ver el resultado principal de $d$-separación e independencia:

\begin{teorema}
    Sean $A$ y $B$ variables de una red bayesiana separadas por $C$, entonces $A \perp B | C$.
\end{teorema}
    
Alternativamente, podemos definir un I-map usando $d$-separación: un DAG sobre un conjunto de variables 
aleatorias es un I-map del modelo de dependencia $M$ si la $d$-separación sobre dos conjuntos de variables 
implica que son independientes.
    
Consecuentemente, una red bayesiana de nodos $X_i$, $i = 1, \ldots , n$ es útil para encontrar probabilidades 
$p(x_1, \ldots, x_n)$, dado que el conjunto de variables satisface la lista de independencias representada por 
el grafo. Esta expresión gráfica de la distribución ha ayudado a desarrollar algoritmos para aplicar algunas 
herramientas de probabilidad como el Teorema de Bayes, o la regla de marginalización. Asimismo, también se 
puede inferir la 
distribución de probabilidad a partir de datos, asumiendo que hay una red bayesiana representando el conjunto 
de independencias entre las variables \cite{buntine}.
    
\subsection{Algoritmos de inferencia para redes bayesianas}
    
La inferencia en redes bayesianas es el método por el cual, dado un conocimiento previo o {\em evidencia}, 
podemos calcular las probabilidades de que ocurran ciertos resultados \cite{pearl88}; dado un conjunto de 
variables aleatorias ${\bf X} = \{X_1, \dots, X_n\}$, ${\bf E} \subset {\bf X}$, con las variables previamente 
conocidas ${\bf E} = \{E_1,\dots, E_m\}$, y un conjunto de valores ${\bf e} = \{e_1, \dots, e_m\}$ tales que 
$e_i \in \Omega_{E_i}$, un {\em algoritmo de inferencia} o {\em algoritmo de propagación} es un método que halla 
el valor de probabilidad 
$$p(x_i|{\bf e}), \quad x_i \in \Omega_{X_i}, \quad \forall X_i \in {\bf X} \backslash {\bf E}.$$

donde el conjunto de variables observadas puede ser vacío.

Si el valor de probabilidad $p(x_i|{\bf e})$ se calcula de una manera exacta, se denomina {\em algoritmo exacto 
de propagación}, mientras que si hacemos una aproximación $\widehat p(x_i|{\bf e})$ de $p(x_i|{\bf e})$, se 
denominará {\em aproximado} \cite{Cano2004}.

El problema de inferencia exacta para el caso general de redes bayesianas es NP-complejo \cite{Cooper}, y 
aunque el de inferencia aproximada también lo sea \cite{Dagum}, estos métodos requieren menos 
carga computacional, y funcionan relativamente bien en casos en los que no es posible hacer inferencia exacta.

Los algoritmos de inferencia exacta \cite {ECastillo, pearl88} están basados en los conceptos de teoría 
de la probabilidad explicados con anterioridad. Entre ellos tenemos el \href{https://ermongroup.github.io/cs228-notes/inference/ve/}{\em 
algoritmo de eliminación de variables} o el \href{https://mjtsai1974.github.io/DevBlog/2018/10/14/bayesian-ml-clique-tree-construction/}{\em 
algoritmo de árbol de unión} \cite{clique-tree}. 

Algunos ejemplos de algoritmos de inferencia aproximada son los 
\href{http://www.sc.ehu.es/sbweb/fisica_/numerico/montecarlo/montecarlo.html}{\em métodos de Monte 
Carlo} \cite{Cano2004}, entre los cuales encontramos el \href{https://github.com/topics/importance-sampling}{\em algoritmo de muestreo de importancia} \cite{martinez}. 

\subsection{Algoritmos de aprendizaje para redes bayesianas}
En este apartado, veremos algunos algoritmos que se pueden usar para obtener redes bayesianas.
\subsubsection{Conceptos}
    
La tarea de aprender una red bayesiana consiste en, dado un conjunto de muestras de la distribución 
conjunta, recuperar la estructura del grafo y el conjunto de distribuciones de probabilidad \cite{different-algorithmic-schemes}.
    
Formalmente, si tenemos un conjunto $({\bf x}^{(1)}, \dots, {\bf x}^{(m)})$ de muestras del conjunto de variables 
${\bf X} = \{X_1, \dots, X_n\}$, nuestro objetivo es encontrar la red bayesiana que mejor representa los datos. Los 
pasos para encontrar una solución óptima son \cite{Cano2004}: 
\begin{enumerate}
    \item Aprendizaje de la estructura de la red.
    \item Aprendizaje del conjunto de parámetros. 
\end{enumerate}

El segundo ha sido muy estudiado, y tiene una solución exacta \cite{neapolitan}. Para el primero, podemos:
\begin{enumerate}
    \item \label{first-scheme} Detectar el conjunto de independencias entre las variables
    \item \label{second-scheme} Buscar una red que represente correctamente las muestras que tenemos.
\end{enumerate}
    

\ref{first-scheme} detecta independencias haciendo tests y añadiéndolas a una lista, para dar una red 
bayesiana que representa a la mayoría o todas las independencias obtenidas de esa manera \cite{learning-algorithms-BN-comparison}, 
mientras que \ref{second-scheme} (llamado ``buscar y puntuar'') busca en el espacio de todas las posibles estructuras 
de redes bayesianas, asignando a cada una un valor real o puntuación con una cierta función o métrica, que mide lo 
adecuada que es la red para esos datos \cite{more-algorithms}.

En ambos casos, el espacio de búsqueda, esto es, el conjunto de todos los DAG posibles entre las $n$ variables, 
es de tamaño {\em híper exponencial}; es un hecho que la búsqueda de la estructura óptima del grafo no se puede 
hacer con un ordenador \cite{Chickering}. Para encontrar una buena estructura de red se usan algoritmos de búsqueda 
aproximada. Entre ellos, aquellos que emplean heurísticas o metaheurística suelen encontrar buenas soluciones. Pasamos 
a ver uno que usa de las heurísticas más fáciles.

\subsubsection{Algoritmo PC}

\subsubsection{Algoritmo de propagación para detrás}
