\chapter{Descripción del problema}

Me gusta jugar al fútbol, estudio informática y matemáticas, con lo que 
surgen de manera natural preguntas en torno a la práctica de este deporte 
alrededor del cual gira todo un mundo, y la importancia de las decisiones 
que se van tomando tanto dentro como fuera del campo.

El análisis del fútbol no es para nada algo nuevo, y no es otra cosa que 
pensar, observar y reflexionar sobre lo que ocurre a lo largo de un partido 
o temporada(s), para actuar consecuentemente y, por ejemplo, fichar a 
ciertos jugadores en detrimento de otros, centrarse más en la defensa 
que en el ataque, entrenar un tipo de jugada o pases específicos, darle 
importancia a la táctica o a la condición física. La lista puede ser muy 
larga, tanto como los factores que, literalmente, entran en juego.

Así pues, este proyecto tiene como idea ayudar desde los jugadores hasta 
el cuerpo técnico del equipo: entrenadores, director técnico, preparadores 
físicos y analistas tácticos (también conocidos como 'scouting').

Dentro del análisis de fútbol, podemos entonces destacar los siguientes roles:

\begin{itemize}
    \item Analista táctico: se encarga de estudiar los equipos y cómo se desenvuelven en los diferentes partidos. Entre sus tareas se incluyen el análisis del propio equipo y de los rivales.
    \item Scouter: es la persona encargada de la búsqueda y captación de jugadores. Su tarea consiste en encontrar los jugadores que necesita el equipo o el club, por lo que analiza sus cualidades y posibilidades de integración al equipo, desde su rendimiento futbolístico hasta el económico.
    \item Analista de datos: toma la información estadística de cada partido tanto en lo individual como en lo colectivo. Establece relaciones para encontrar respuestas o ideas que puedan colaborar con la toma de decisiones del club referentes a lo táctico, lo técnico y lo económico, para el modelo de juego del equipo o para la compra y venta de jugadores.
\end{itemize}
Manteniendo esto en mente, nuestra solución 
podría ser usada por ellos. Vemos entonces 
que el trabajo se puede enfocar de diversas 
maneras, y conforme avancemos habremos de descartar algunas y 
quedarnos con otras, quedando todo debidamente justificado. Dependerá de los 
datos que se encuentren disponibles, y la necesidad que haya en el mercado. 
Normalmente y si consultamos la literatura, los análisis son estáticos; 
rendimiento y mapas de calor de un jugador, desde dónde se ha lanzado más 
a portería, etc. pero no hay tanto estudiado en cuanto a causalidad.

Llevándolo un poco a tierra y como motivación principal del presente proyecto,
\textbf{mi cliente como jugador aficionado me ha planteado muchas veces por qué 
 no se usa algún tipo de análisis para, dadas las personas que se presentan 
 a un partido, hacer rápidamente unos equipos que estén equilibrados, y en 
 los que cada uno tenga una posición en la que estén cómodos y asegure un 
 buen resultado, así como que los cambios estén claros y se puedan hacer 
 sin perder tiempo.}
