\chapter{Descripción del problema}

Este proyecto tiene como idea ayudar desde los jugadores hasta 
el cuerpo técnico del equipo: entrenadores, director técnico, preparadores 
físicos y analistas tácticos,también conocidos como 'scouting'.

\section{Metodología - Mentalidad ágil}
Lo esencial de esta forma de proceder es que se centra en resolver problemas, 
no en hacer aplicaciones: importa el por qué, antes que el qué o el cómo. Estos 
últimos serán resultado de ese primer análisis y empatización con el cliente, 
como iremos viendo en este apartado. 

Así pues, el punto de partida es pensar la motivación o problema que queremos resolver. 
¿Por qué queremos hacer este TFG? ¿A quién ayuda? ¿Quién lo usaría? ¿Qué solución proponemos? 
Los problemas deben estar antes que nada; es complicado comprar ingredientes si no sabes qué 
receta vas a preparar. 

Luego, de la motivación saldrán los objetivos que nos planteamos. Estos habrán de indicar 
qué es lo que se quiere conseguir, incluyendo qué tipo de medios tenemos disponibles 
para resolver el problema.





y de los objetivos 
una serie de productos mínimamente viables. 


\section{Clientes}
A su vez, dentro del análisis de fútbol, podemos destacar los siguientes roles:

\begin{itemize}
    \item Analista táctico: estudia los equipos y cómo se desenvuelven en los partidos. 
    Querrá obtener un análisis del propio equipo y de los rivales.
    \item Scouter: es la persona encargada de la búsqueda y captación 
    de jugadores. Querrá encontrar los jugadores que necesita el equipo 
    o el club, por lo que analiza sus cualidades y posibilidades de 
    integración al equipo, desde su rendimiento futbolístico hasta el 
    económico.
    \item Analista de datos: toma la información estadística de 
    cada partido tanto en lo individual como en lo colectivo. 
    Establece relaciones para encontrar respuestas o ideas que 
    puedan colaborar con la toma de decisiones del club referentes 
    a lo táctico, lo técnico y lo económico, para el modelo de 
    juego del equipo o para la compra y venta de jugadores. Por ejemplo: del 
    equipo femenino de Noruega. 
    \item Persona que apuesta: querrá inferir el resultado de un partido, 
    quién marcará y así, una vez se publiquen los jugadores. 
    \item Entrenador: querrá saber a quién sacar durante un partido, los 
    cambios, dónde poner a quién. Por ejemplo: entrenadora principal y 
    segunda entrenadora del Club de Fútbol Internacional del Granada.
    \item Matemático 
    \item Informático 
    \item Aficionado  
\end{itemize}


\section{Historias de usuario}  



\section{Métricas en el fútbol moderno: concepto, análisis y uso}

\subsection{Expected goals o Goles esperados, xG}


\subsection{Expected assists o Asistencias esperadas, xA}


\subsection{Expected points o Puntos esperados,xP o xPts}


\subsection{Pases Permitidos por Acción Defensiva,PPDA}



