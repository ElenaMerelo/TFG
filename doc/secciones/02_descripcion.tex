\chapter{Descripción del problema}
El presente trabajo intentará responder a: ¿La entropía mejora las posibilidades 
de marcar gol? Si cambia la entropía del equipo, ¿podemos determinar la causa? Esto es, 
si ha sido por un jugador específico, la alineación o más bien la influencia del equipo 
contrario. Asimismo, ¿es posible ver la entropía reflejada en un tipo de visualización de 
las redes de pases?. Estudiaremos pues hasta qué punto es determinante la 
entropía a nivel de jugador o equipo, empleando técnicas estadísticas, para obtener respuestas a lo planteado.

\section{Metodología - Mentalidad ágil}
La metodología nos dicta dos cosas esenciales:
\begin{enumerate}
    \item ¿Qué tengo que hacer ahora?
    \item ¿Lo que he hecho está bien y era lo que tenía que hacer?
\end{enumerate}

Para el primer punto, hacemos uso de \textit{issues} y \textit{milestones}, 
como veremos más adelante. Para lo segundo, hay que tener en cuenta que todos 
los \textit{issues} son problemas, e idealmente deben de decir explícitamente 
(o implícitamente si está claro) cómo se resuelve el problema.

La mentalidad ágil tiene su origen en el manifesto ágil\cite{manifesto-agil}, el cual fue una revolución frente 
al modelo en cascada,
en la forma de desarrollar software, y planteaba que tienen más valor las personas que 
los procesos o herramientas, un producto funcionando que mucha documentación, colaboración 
con los clientes que contratos, y \textbf{flexibilidad} frente a seguir un plan. Desde el 
año 2001 que se redactó ha cambiado mucho la tecnología, y por ejemplo nosotros pondremos 
mucho peso en la documentación, mas la idea sigue intacta: lo importante son los usuarios, 
adaptarse. En general pues, con \textit{ágil} nos referimos a una forma de pensar que se aplica 
a todo un ciclo de desarrollo del software centrado en el cliente y que consiste en 
continuas mejoras de cada vez más complejos productos mínimamente viables\cite{agile-science}.
En este apartado entenderemos mejor y dejaremos claro lo que esto significa e implica.

Lo esencial de esta forma de proceder es que su objetivo es \textbf{resolver problemas} y satisfacer 
al cliente, no hacer aplicaciones: importa el por qué, antes que el qué o el cómo. Estos 
últimos resultarán de ese primer análisis, seguidos de una empatización, que consiste 
en contactar con clientes, leer prensa y demás para enfocar el problema; posteriormente 
ideación, de la que saldrán los objetivo e hitos o historias de usuario; y por último diseño, 
que será llevar a tierra todo lo anterior. 

Asimismo, el punto de partida es pensar la motivación o problema que queremos resolver. 
¿Por qué queremos hacer este TFG? ¿A quién ayuda? ¿Quién lo usaría? ¿Qué solución proponemos? 
Los problemas deben estar antes que nada; es complicado comprar ingredientes si no sabes qué 
receta vas a preparar. 

Luego, de la motivación saldrán los objetivos que nos planteamos. Estos habrán de indicar 
qué es lo que se quiere conseguir, incluyendo qué tipo de medios tenemos disponibles. Lo 
más importante es que han de estar en el dominio del problema, tienen que ser específicos, 
medibles y alcanzables\cite{objetivos}. 

De estos objetivos saldrán una serie de productos mínimamente viables. 

Las historias de usuario sirven para centrarnos en los problemas que queremos solucionar 
y los objetivos a alcanzar. Están relacionadas con la lógica de 
negocio del proyecto y siempre son un beneficio para el los posibles usuarios del proyecto.

Una regla del pulgar para las historias de usuario: Siempre tienen que expresar un deseo y un 
beneficio para el usuario. Si pones "ojalá qué" y te lo imaginas en la boca del usuario y suena 
creíble, es que es una historia de usuario. Si no, pues no.
"Ojalá que pueda iniciar sesión y registrarme" ¿Suena creíble? ¿No? Pues no es una historia 
de usuario, es un issue o tarea que tú necesitas que el usuario haga para que cumpla sus deseos.


El problema es poner lo que tú quieres que haga el usuario para conseguir algo, y no lo que el 
usuario quiere. Está claro que nosotros no tenemos clientes (en la mayoría de los casos, aunque 
os animo que busquéis a "stakeholders" que os puedan dar pistas) pero un usuario no quiere 
logearse o quieren las cosas en React. Quiere tener claro cuantos pacientes tiene que atender 
para optimizar su tiempo.

Por otro lado, los issues  plantean un problema. Siempre están enmarcados en un \textit{milestone}, 
y tienen que tener un criterio de aceptación para ser cerrados. Hacer tareas lo más atómicas posibles ayuda, porque 
se hace un PR, se termina una tarea, y se avanza menos a trompicones.

Todo el código se incorpora mediante PRs

Un \textit{milestone} describe un producto mínimamente viable, y el estado en el que tiene 
que estar el repositorio al terminar, además de los criterios que se daben seguir 
para validarlo.

Dentro de la metodología descrita, nos enmarcaremos en el \href{https://www.iebschool.com/blog/design-thinking-agile-scrum/}{design thinking}, 
un proceso iterativo y no lineal que consta de una fase inicial consistente en empatizar con el 
cliente, para saber qué necesita. Seguidamente, se define el problema, se piensa en una solución al 
mismo, se desarrolla un prototipo, y finalmente se testea. 


\section{Herramientas usadas}
En esta sección profundizaremos un poco en las herramientas empleadas para desarrollar el trabajo, 
y por qué las hemos escogido.
% issue #53 
\subsection{\href{https://github.com/}{Github}} 

Es una base de datos online que permite llevar cuenta de los proyectos de Git, fuera del ordenador o 
servidor local, y está basado en la nube. Git es un sistema distribuido de control de versiones, gratis y 
cuyo software es libre. Es fácil de aprender y tiene un gran rendimiento.

Así pues, Github es una página web muy completa para que desarrolladores y programadores puedan trabajar colaborativamente 
en repositorios. Su punto fuerte es el sistema de control de versiones, que permite llevar cuenta detallada 
de los cambios realizados por cada persona, discutirlos, revisarlos y proponer modificaciones, así como 
separar el producto final de las funcionalidades que se vayan añadiendo y sobre las que cada equipo esté 
trabajando. Adicionalmente, proporciona herramientas como las \href{https://github.com/features/actions}{Github Actions}, 
las cuales facilitan la automatización de los flujos de trabajo, incluyendo integración y despliegue contínuos.
Nosotros lo empleamos para chequear la ortografía cada vez que subimos algo al repositorio, o construir el pdf 
de la memoria cuando se cambie algún archivo \textit{.tex}, pero por supuesto hay muchísimas más posibilidades.


\subsection{Oh my zsh}
Para tener conciencia situacional del estado del repositorio, instalé \href{https://ohmyz.sh/}{oh my zsh}, 
un \textit{framework} libre para configurar \textit{Zsh}. Permite el acceso y fácil instalación de plantillas para 
la terminal, que hacen más fácil saber en qué rama se está trabajando, qué cambios se han realizado, y en 
general casa genial con \textit{Github} y sus funcionalidades. Tiene detrás una comunidad muy grande de 
contribuidores, incluye numerosos \textit{plugins} que hacen el desarrollo de \textit{software} más fácil, 
y la posibilidad de personalizar la terminal, con temas ya creados o propios.

{Zsh} es un \textit{shell} que se presenta como alternativa a \textit{bash}, el cual viene por defecto en 
\textit{Ubuntu}. Las ventajas de las que más nos hemos aprovechado y por las que lo escogimos parten de 
su mayor configurabilidad, el que te corrija errores de escritura y complete palabras.

\section{Clientes}
Basándonos en la \href{https://www.designthinking.services/herramientas-design-thinking/metodo-persona/}{metodología basada en personas}, 
hemos llegado a los siguientes usuarios: 
\begin{itemize}
    \item Analista táctico: estudia los equipos y cómo se desenvuelven en los partidos. 
    Querrá obtener un análisis del propio equipo y de los rivales.
    \item Scouter: es la persona encargada de la búsqueda y captación 
    de jugadores. Querrá encontrar los jugadores que necesita el equipo 
    o el club, por lo que analiza sus cualidades y posibilidades de 
    integración al equipo, desde su rendimiento futbolístico hasta el 
    económico.
    \item Analista de datos: toma la información estadística de 
    cada partido tanto en lo individual como en lo colectivo. 
    Establece relaciones para encontrar respuestas o ideas que 
    puedan colaborar con la toma de decisiones del club referentes 
    a lo táctico, lo técnico y lo económico, para el modelo de 
    juego del equipo o para la compra y venta de jugadores. Por ejemplo: del 
    equipo femenino de Noruega. 
    \item Persona que apuesta: querrá inferir el resultado de un partido y 
    quién marcará, una vez se publiquen los jugadores. 
    \item Entrenador: querrá saber a quién sacar durante un partido, los 
    cambios, dónde poner a quién. Por ejemplo: entrenadora principal y 
    segunda entrenadora del Club de Fútbol Internacional del Granada.  
    \item Periodista deportivo: querrá tener una herramienta más en su arsenal, y 
    usarla para obtener gráficos, estadísticas, y conclusiones acerca de un partido, 
    temporada, equipo o jugador.
\end{itemize}


\section{Historias de usuario}  

A partir de los clientes, y como parte de la metodología ya explicada, definimos en \href{https://github.com/ElenaMerelo/TFG}{mi repositorio 
de GitHub} las siguientes historias de usuario: 

\subsection{Como tutores, nos gustaría que el proyecto siga un desarrollo ágil}

En la línea de lo que hemos explicitado con anterioridad, el trabajo estará orientado 
al cliente.
Como tutores, nos gustaría que el proyecto use un desarrollo ágil, orientado al cliente, 
que me lleve a montar la infraestructura del TFG, para después empatizar con las personas 
a las que va dirigido, idear y diseñar el resto.

\subsection*{Como matemática, quiero que la teoría sobre redes causales se aplique bien al caso concreto}

Y que se conecte sin problema con la parte informática.

\section{Milestones o Productos Mínimamente Viables}


\section{Diccionario de términos}
% issue #50

\subsection{Entropía}

\subsection{Entropía de Tsallis}

\subsection{Entropía de Shannon}

\subsection{Entropía de Renyi} 


\subsection{Principio de máxima entropía}

\subsection{Redes causales}