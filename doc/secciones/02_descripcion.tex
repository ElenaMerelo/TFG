\chapter{Descripción del problema}

Este proyecto tiene como idea ayudar desde los jugadores hasta 
el cuerpo técnico del equipo: entrenadores, director técnico, preparadores 
físicos y analistas tácticos,también conocidos como 'scouting'.

\section{Perfiles dentro del análisis de fútbol}

scout, analista, ojeador 


\section{Métricas en el fútbol moderno: concepto, análisis y uso}

\subsection{Expected goals o Goles esperados, xG}


\subsection{Expected assists o Asistencias esperadas, xA}


\subsection{Expected points o Puntos esperados,xP o xPts}


\subsection{Pases Permitidos por Acción Defensiva,PPDA}


\section{Metodología}


\section{Clientes}
A su vez, dentro del análisis de fútbol, podemos destacar los siguientes roles:

\begin{itemize}
    \item Analista táctico: se encarga de estudiar los equipos y cómo 
    se desenvuelven en los diferentes partidos. Querrá obtener un análisis 
    del propio equipo y de los rivales.
    \item Scouter: es la persona encargada de la búsqueda y captación 
    de jugadores. Querrá encontrar los jugadores que necesita el equipo 
    o el club, por lo que analiza sus cualidades y posibilidades de 
    integración al equipo, desde su rendimiento futbolístico hasta el 
    económico.
    \item Analista de datos: toma la información estadística de 
    cada partido tanto en lo individual como en lo colectivo. 
    Establece relaciones para encontrar respuestas o ideas que 
    puedan colaborar con la toma de decisiones del club referentes 
    a lo táctico, lo técnico y lo económico, para el modelo de 
    juego del equipo o para la compra y venta de jugadores. Por ejemplo: del 
    equipo femenino de Noruega. 
    \item Persona que apuesta: querrá inferir el resultado de un partido, 
    quién marcará y así, una vez se publiquen los jugadores. 
    \item Entrenador: querrá saber a quién sacar durante un partido, los 
    cambios, dónde poner a quién. Por ejemplo: entrenadora principal y 
    segunda entrenadora del Club de Fútbol Internacional del Granada.
    \item Matemático ?
    \item Informático ?
    \item Aficionado  ?
\end{itemize}

Manteniendo esto en mente, nuestra solución 
podría ser usada por ellos. Vemos entonces 
que el trabajo se puede enfocar de diversas 
maneras, y conforme avancemos habremos de descartar algunas y 
quedarnos con otras, quedando todo debidamente justificado. Dependerá de los 
datos que se encuentren disponibles, y la necesidad que haya en el mercado. 
Normalmente y si consultamos la literatura, los análisis son estáticos; 
rendimiento y mapas de calor de un jugador, desde dónde se ha lanzado más 
a portería, etc. pero no hay tanto estudiado en cuanto a causalidad.

\section{Historias de usuario}  


