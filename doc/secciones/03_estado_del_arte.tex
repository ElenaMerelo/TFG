\chapter{Estado del arte}

En los últimos años ha habido una revolución en el análisis de la organización 
y rendimiento de los equipos de fútbol y sus jugadores, por lo que es posible tener acceso a todos los 
eventos que ocurren en el campo, tales como pases, disparos o goles, todo ello con las coordenadas 
exactas de tiempo y posición, y el jugador responsable de cada evento. Por otro lado, también es 
posible llevar un registro de las posiciones de todos los jugadores en el campo, junto con el balón, 
lo que permite determinar la posición, velocidad y aceleración de cada jugador, y dando información 
muy útil sobre su desempeño físico y táctico. 

No obstante, los avances más determinantes e importantes tienen que ver con la posibilidad de aplicar o 
definir nuevos métodos y herramientas. En ese sentido, se pueden entender los roles de los jugadores 
como un todo, no solo como componentes aisladas sin interacciones entre ellos. Consecuentemente, es 
posible contruir redes de pases compuestas por nodos, correspondientes a los jugadores, y enlaces, 
que representan los pases entre ellos. Esta organización está además lejos de ser 
aleatoria. El análisis de la evolución de las redes de pases ha mostrado que sus propiedades 
cambian continuamente a lo largo de un partido y que eventos importantes tales como goles 
pueden afectar la organización de la red.\cite{spatial-and-temporal-entropies}

Asimismo, durante la última década las redes bayesianas se han popularizado en el campo 
de la inteligencia artificial, y hay numerosos estudios que buscan predecir los resultados 
de partidos de fútbol empleándolas, para lo que construyen diferentes modelos\cite{prediction-barcelona}.
En \cite{razali-2017} consiguen una precisión predictiva del 75.09\%. \cite{dolores} introduce el uso 
de calificaciones dinámicas y redes bayesianas híbridas para hacer una predicción del resultado de un 
partido entre dos equipos \textit{a} y \textit{b} a partir de datos históricos de partidos 
en los que no participan ni \textit{a} ni \textit{b}. Ello nos muestra 
el increíble potencial que tienen estas redes también, y no solo las más tradicionales 
redes complejas.

Adicionalmente, hay estudios\cite{smart-data} que ponen un gran énfasis 
en aplicar conocimiento causal al proceso de desarrollo del modelo, basándose 
en los datos que son necesarios para la predicción. De esa manera, consiguen predicciones precisas 
del cambiante rendimiento de equipos de fútbol. El modelo permite predecir, antes de que empiece 
una temporada, los puntos totales en la liga que se espera un equipo acumule a lo largo de la temporada, lo 
que supone un cambio de perspectiva con respecto a los artículos mencionados con anterioridad, y en su 
metodología construye y trabaja sobre la literatura anterior, extendiéndola.

Por otro lado, hay vertientes que hacen análisis de fútbol desde la inferencia estadística, desde 
el \textit{machine learning}, o ambos\cite{ML-inference}. En este \textit{paper}, muestran que 
los equipos están caracterizados por dónde en el campo realizan pases, y se 
pueden identificar por la manera en que pasan el balón. Usando mapas de calor de las localizaciones 
de los pases, consiguen una precisión del 87\% en una clasificación de veinte equipos. Emplean 
además la localización de los pases a lo largo de una posesión del balón para predecir tiros a 
puerta. Finalmente, usan los pesos del modelo de predicción para categorizar a los jugadores 
según el valor de sus pases. Nos muestra una vez más la diversidad que existe en el estudio 
y análisis del fútbol; no hay límites en cuanto a lo que se puede considerar y obtener.

En cuanto a entropía, \cite{entropy-analysis}, afirma que para optimizar su rendimiento, un 
equipo debe mantener un equilibrio entre orden u organización, el cual 
propicia la cooperación entre sus miembros, y desorden o desorganización, que confunde 
al oponente y favorece el mantener un cierto grado de libertad. Para medir, usan entropía de 
Tsallis a partir de los pases entre los jugadores, y concluyen que la posición de un equipo 
al final de la temporada está correlada con la entropía del mismo. 

Más específicamente, saber quién pasa el balón a quién significa reducir la entropía de los pases y maximizar 
la comunicación y cooperación entre los jugadores del equipo. A su vez, esto implicaría limitar sus 
grados de libertad al mover la pelota y sorprender al oponente, lo que expondría al equipo a contraataques. El precio 
pues de maximizar la certeza de los pases puede ser un comportamiento predecible y hacer al equipo vulnerable. Lo más importante 
de este \textit{paper} es que reivindica que el rendimiento de un equipo de fútbol puede 
ser modelado usando la entropía de sus pases de balón.

Y no solo eso, \cite{spatial-and-temporal-entropies} cuantifica la entropía espacial y temporal de 
equipos de fútbol y sus jugadores, también mediante interacciones basadas en pases. Sus resultados 
muestran que la entropía espacial cambia de acuerdo con la posición de los jugadores en el campo, y la 
organización de las redes de pases cambian a lo largo de un partido. 

\cite{network-analysis} fue uno de los primeros en investigar la entropía entre los jugadores de un equipo de 
fútbol. Analizan el desempeño y estilo de juego de la selección española en los mundiales del 2010 desde una 
perspectiva temporal, con lo que medidas globales como el número de pases consecutivos o el número de 
pases por minuto reflejan el éxito de un equipo en imponer su forma de juego. Observando el coeficiente 
de agrupación se captura la naturaleza combinatoria del estilo de pases, habitualmente hechos a corta distancia.
Así, España consigue tener mayor posesión del balón y pases completados, lo que es una buena estrategia defensiva 
al privar al otro equipo de influenciar en el juego. Esto en último lugar determina el resultado final del partido, 
si bien es discutible la necesidad de tener tanto tiempo el balón, sin muchas veces avanzar a puerta. En el 
último partido muestran que la precisión disminuye y el juego es menos elaborado, al introducirse 
más emoción y nervios.

Pero, ¿cómo medimos la calidad de un equipo? No hay muchos artículos aplicando redes causales a partidos de 
fútbol. \cite{cerqueira} muestra que las estrategias ofensivas son más influyentes que las defensivas.


% #51 


 