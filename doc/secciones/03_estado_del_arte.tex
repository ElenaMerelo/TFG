\chapter{Estado del arte}
% añadir en parte matemática explicación hybrid BN y dynamic ratings
% explicar en parte matemática qué es entropía de Tsallis y super-additive entropy index
% explicar entropía de shannon, de renyi, decir por qué nosotros usaremos la de tsallis 
% mencionar interaction performance theory, asbhys law of requisite variety and jaynes 
% principle of maximum entropy, permutation entropy
En este apartado veremos trabajos relacionados que nos han parecido interesantes, 
aportan algo a nuestro estudio, o nos ayudaron a encaminarnos. El primero que más 
llama la atención es \textbf{Dolores: un modelo que predice resultados de partidos 
de fútbol de todo el mundo}\cite{dolores}, el cual, usando calificaciones dinámicas 
y redes bayesianas híbridas, hace una predicción del resultado de un partido entre dos 
equipos  \textit{a} y \textit{b} a partir de datos históricos de partidos 
en los que no participan ni \textit{a} ni \textit{b}. Nosotros nos 
centraremos en redes bayesianas normales, y no usaremos entrenamiento de datos o 
\textit{machine learning}, pero es bueno tener este estudio en mente, al mostrarnos 
el increíble potencial que tienen estas redes también, y no solo las más tradicionales 
redes complejas.

De \textbf{Predicción de resultados de fútbol con redes bayesianas en la
Liga Española-Equipo Barcelona}\cite{prediction-barcelona} destaca el uso de 
\textit{NETICA software}\cite{netica}, el cual construye las redes usando un sistema de 
preguntas y respuestas. Sin embargo, nosotros no lo usaremos 
debido a que no se adapta completamente a nuestro problema. Como este estudio hay 
numerosos más, por ejemplo \textbf{Predicción de resultados de partidos de fútbol 
usando redes bayesianas en la Premier League inglesa}\cite{razali-2017}, 
cuyas redes consiguieron una precisión predictiva del 75.09\%, empleando datos de 
tres temporadas.

Adicionalmente, \textbf{Hacia datos inteligentes: Mejorando la precisión predictiva 
en rendimiento a largo plazo de un equipo de fútbol}\cite{smart-data} pone un gran énfasis 
en aplicar conocimiento causal al proceso de desarrollo del modelo, basándose 
en los datos que son necesarios para la predicción. De esa manera, desarrollan un modelo 
que genera predicciones precisas del cambiante rendimiento de equipos de fútbol. El 
modelo permite predecir, antes de que empiece una temporada, los puntos totales en 
la liga que se espera un equipo acumule a lo largo de la temporada. Lo 
destacamos por el hecho de que construye y trabaja sobre toda la literatura anterior, y la extiende 
de una manera muy completa.

Incluimos \textbf{Usando \textit{machine learning} para obtener inferencias sobre los datos de 
ubicación de pases en el fútbol}\cite{ML-inference} por su uso de la inferencia estadística. 
En este \textit{paper}, introducen dos maneras de analizar datos de eventos pasados para 
revelar eventos no tan obvios. Aplican sus métodos a datos de la temporada 2012-2013 de La Liga 
y curiosamente muestran que los equipos están caracterizados por dónde en el campo realizan pases, y se 
pueden identificar por la manera en que pasan el balón. Usando mapas de calor de las localizaciones 
de los pases, consiguen una precisión del 87\% en una clasificación de veinte equipos. Emplean 
además la localización de los pases a lo largo de una posesión del balón para predecir tiros a 
puerta. Finalmente, usan los pesos del modelo de predicción para categorizar a los jugadores 
según el valor de sus pases. Nos muestra una vez más la diversidad que existe en el estudio 
y análisis del fútbol; no hay límites en cuanto a lo que se puede considerar y obtener.

Asimismo, merece mucho la pena mencionar \textbf{El comportamiento adaptativo de un equipo 
de fútbol: un análisis basado en la entropía}\cite{entropy-analysis}, dado que guarda una 
alta relación con lo que nosotros queremos estudiar y nos ocupa. A grandes rasgos, afirma que 
para optimizar su rendimiento, un equipo debe mantener un equilibrio entre orden u organización, el cual 
propicia la cooperación entre sus miembros, y desorden o desorganización, que confunde 
al oponente y favorece el mantener un cierto grado de libertad. Para medir, usan entropía de 
Tsallis a partir de los pases entre los jugadores, y concluyen que la posición de un equipo al final de la temporada está 
correlada con la entropía del mismo. 

Más específicamente, saber quién pasa el balón a quién significa reducir la entropía de los pases y maximizar 
la comunicación y cooperación entre los jugadores del equipo. A su vez, esto implicaría limitar sus 
grados de libertad al mover la pelota y sorprender al oponente, lo que expondría al equipo a contraataques. El precio 
pues de maximizar la certeza de los pases puede ser un comportamiento predecible y hacer al equipo vulnerable. Lo más importante 
de este \textit{paper} para nosotros es que reivindica que el rendimiento de un equipo de fútbol puede 
ser modelado usando la entropía de sus pases de balón, lo que nos confirma que estamos siguiendo el camino correcto.

Son muy destacables y nos han inspirado las contribuciones realizadas por Anthony 
Constantinou\cite{a-constantinou}, cuya investigación se encuentra en el campo 
de las redes bayesianas, ha desarrollado varios modelos basados en ellas, y publicado 
numerosos artículos aplicándolos al análisis de fútbol. 

Dejamos para lo último el artículo que más encaja con nuestro trabajo,
\textbf{Entropías espaciales y temporales en la liga española de fútbol}\cite{spatial-and-temporal-entropies}. 
En él, cuantifican la entropía espacial y temporal de equipos de fútbol y sus jugadores, 
mediante interacciones basadas en pases. Sus resultados muestran que la entropía 
espacial cambia de acuerdo con la posición de los jugadores en el campo, y la 
organización de las redes de pases cambian a lo largo de un partido. 

Aprovecha que en los últimos años ha habido una revolución en el análisis de la organización 
y rendimiento de los equipos de fútbol y sus jugadores, por lo que es posible tener acceso a todos los 
eventos que ocurren en el campo, tales como pases, disparos o goles.
Todo ello con las coordenadas exactas de tiempo y posición, y el jugador responsable de cada 
evento. Por otro lado, también es posible llevar un registro de las posiciones de todos los 
jugadores en el campo, junto con el balón, lo que 
permite determinar la posición, velocidad y aceleración de cada jugador, y dando información 
muy útil sobre su desempeño físico y táctico.

No obstante, los avances más determinantes ha sido la posibilidad de aplicar o definir nuevos métodos 
y herramientas. En ese sentido, se puede 
entender los roles de los jugadores como un todo, no solo como componentes aisladas sin 
interacciones entre ellos. Consecuentemente, es posible contruir redes de pases, compuestas 
por nodos (jugadores), y enlaces (pases entre ellos), cuya organización está lejos de ser 
aleatoria. El análisis de la evolución de las redes de pases ha mostrado que sus propiedades 
cambian continuamente a lo largo de un partido y que eventos importantes tales como goles 
pueden afectar la organización de la red.

Uno de los primeros en investigar la entropía entre los jugadores de un equipo de 
fútbol fue \textbf{Un análisis de las redes de juego del equipo campeón de la copa 
mundial FIFA 2010}\cite{network-analysis}. Analizan el desempeño y estilo de juego de la selección española en los mundiales del 2010 desde una 
perspectiva temporal, con lo que medidas globales como el número de pases consecutivos o el número de 
pases por minuto reflejan el éxito de un equipo en imponer su forma de juego. Observando el coeficiente 
de agrupación se captura la naturaleza combinatoria del estilo de pases, habitualmente hechos a corta distancia.
Así, España consigue tener mayor posesión del balón y pases completados, lo que es una buena estrategia defensiva 
al privar al otro equipo de influenciar en el juego. Esto finalmente determina el resultado final del partido, 
si bien es discutible la necesidad de tener tanto tiempo el balón, sin muchas veces avanzar tanto a puerta. Es 
curioso cómo en el último partido muestran que la precisión disminuye y el juego es menos elaborado, al introducirse 
más emoción y nervios.


 