\chapter{Estado del arte}
% añadir en parte matemática explicación hybrid BN y dynamic ratings
En este apartado veremos trabajos relacionados que nos han parecido interesantes, 
aportan algo a nuestro estudio, o nos ayudaron a encaminarnos. El primero que más 
llama la atención es \textbf{Dolores: un modelo que predice resultados de partidos 
de fútbol de todo el mundo}\cite{dolores}, el cual, usando calificaciones dinámicas 
y redes bayesianas híbridas, hace una predicción del resultado de un partido entre dos 
equipos  \textit{a} y \textit{b} a partir de datos históricos de partidos 
en los que no participan ni \textit{a} ni \textit{b}. Nosotros en un principio nos 
centraremos en redes bayesianas normales, y no usaremos entrenamiento de datos o 
\textit{machine learning}, pero es bueno tener este estudio en mente, al mostrarnos 
el increíble potencial que tienen estas redes también, y no solo las más tradicionalmente 
usadas redes complejas.

De  \textbf{Predicción de resultados de fútbol con redes bayesianas en la
Liga Española-Equipo Barcelona}\cite{prediction-barcelona} destaca el uso de 
\textit{NETICA software}\cite{netica}, el cual construye las redes usando un sistema de 
preguntas y respuestas, dando la posibilidad de introducir información relevante, y 
finalmente proveyendo al usuario de conclusiones. Sin embargo, nosotros no lo usaremos 
debido a que no se adapta completamente a nuestro problema. Como este estudio hay 
numerosos más, por ejemplo \textbf{Predicción de resultados de partidos de fútbol 
usando redes bayesianas en la Premier League inglesa}\cite{razali-2017}, 
cuyas redes consiguieron una precisión predictiva del 75.09\%, empleando datos de 
tres temporadas.

Adicionalmente, \textbf{Hacia datos inteligentes: Mejorando la precisión predictiva 
en rendimiento a largo plazo de un equipo de fútbol}\cite{smart-data} pone un gran énfasis 
en aplicar conocimiento causal y hechos reales al proceso de desarrollo del modelo, basándose 
en los datos que son necesarios para la predicción, en vez de los que hay disponibles.
De esa manera, desarrollan un modelo que genera predicciones precisas del cambiante rendimiento 
de equipos de fútbol, usando datos limitados. El modelo permite predecir, antes de que 
empiece una temporada, los puntos totales en la liga que se espera un equipo acumule 
a lo largo de la temporada. Los resultados se comparan bien con otros tipos de 
modelos relevantes, y obtiene resultados en línea con modelos que usan muchos más datos.

% explicar en parte matemática qué es entropía de Tsallis y super-additive entropy index
% explicar entropía de shannon, de renyi, decir por qué nosotros usaremos la de tsallis 
% mencionar interaction performance theory, asbhys law of requisite variety and jaynes principle of maximum entropy 
Asimismo, merece mucho la pena mencionar \textbf{El comportamiento adaptativo de un equipo 
de fútbol: un análisis basado en la entropía}\cite{entropy-analysis}, dado que guarda una 
alta relación con lo que nosotros queremos estudiar y nos ocupa. A grandes rasgos, afirma que 
para optimizar su rendimiento, un equipo debe mantener un equilibrio entre orden u organización, el cual 
propicia la cooperación entre sus miembros, y desorden o desorganización, que confunde 
al oponente y favorece el mantener un cierto grado de libertad. Para medir, usan entropía de 
Tsallis a partir de los pases entre los jugadores, y concluyen que la posición de un equipo al final de la temporada está 
correlada con la entropía del mismo, medida con un índice de entropía súper-aditivo. 

Más específicamente, saber quién pasa el balón a quién significa reducir la entropía de los pases y maximizar 
la comunicación y cooperación entre los jugadores del equipo. A su vez, esto implicaría limitar sus 
grados de libertad al mover la pelota y sorprender al oponente, lo que expondría al equipo a contraataques. El precio 
pues de maximizar la certeza de los pases puede ser un comportamiento predecible y hacer al equipo vulnerable. Lo más importante 
de este \textit{paper} para nosotros es que reivindica que el rendimiento de un equipo de fútbol puede 
ser modelado usando la entropía de sus pases de balón, lo que nos confirma que estamos siguiendo el camino correcto.

% me quedan paper de jjmerelo, el de jm buldú, y pdf de cerqueira
% justificar y terminar de encajar todo con lo mío
Son muy destacables y nos han inspirado las contribuciones realizadas por Anthony 
Constantinou\cite{a-constantinou}, cuya investigación se encuentra en el campo 
de las redes bayesianas, ha desarrollado varios modelos basados en ellas, y publicado 
numerosos artículos aplicándolos al análisis de fútbol. 