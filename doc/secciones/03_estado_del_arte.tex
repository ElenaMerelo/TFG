\chapter{Estado del arte}
La historia del análisis de fútbol comienza en 1950, cuando Charles Reep \cite{reep-bio}, teniente coronel de 
la \textit{Royal Air Force} británica y contable, empezó a 
registrar sistemáticamente los eventos que tenían lugar a lo largo de un partido de Swindon Town. Con ello, no pretendía 
únicamente tener detalle de lo que había ocurrido, sino también saber por qué pasaba y qué podían aprender los 
equipos de los datos, para mejorar su juego. Tras décadas, llegó a la conclusión de que el mayor número 
de goles resultaba de posesiones de tres pases o menos, por lo que aludía que los equipos debían simplificar 
sus tácticas para llegar a portería más rápido.

El problema de Reep no fue su recolección de datos, sino su análisis; podría haberse preguntado sobre 
la tasa de goles en posesiones de varias distancias, o cómo se desarrollaron exactamente esas 
posesiones de tres pases. 

Otro pionero en el uso de datos en el deporte fue \href{http://www.riazhaq.com/2022/08/remembering-salam-qureishi-pillar-of.html}{Abdus Salam Qureishi}, 
filántropo e informático en Sillicon Valley, hacía \textit{datascouting} en la reconstrucción de los 
\href{https://www.dallascowboys.com/}{Dallas Cowboys}, un equipo profesional de fútbol americano de los años 60 \cite{chazan2020sports}.

En los años 70, \href{https://es.wikipedia.org/wiki/Valeri_Lobanovski}{Valeri Lobanovski} fue un jugador y entrenador de 
fútbol ucraniano que formalizó el cuerpo técnico moderno y la captura de eventos en el fútbol del equipo \href{https://es.wikipedia.org/wiki/F._C._Dinamo_de_Kiev}{Dinamo de 
Kiev}. Declaraba que ``Hoy en día los jugadores no pueden quejarse, saben que mañana después del partido habrá colgada una hoja con 
las cifras que describen en detalle su juego" \cite{kilpatrick2011inverting}.

En 1996, \href{https://es.wikipedia.org/wiki/Opta_Sports}{Opta} empezó a acumular datos de partidos para la 
\textit{Premier League} inglesa. Ello incluía todas las estadísticas a las que un aficionado al fútbol está 
acostumbrado: número de pases, número de regateos, distancia recorrida, por ejemplo. Este podría considerarse 
el punto de comienzo del análisis de fútbol moderno. Desde 2019, se llama \href{https://www.statsperform.com/}{Stats Perform}, y 
es todo un referente a la hora de análisis de datos de deportes, incorporando inteligencia artificial.

En 2003, Michael Lewis lanzó el libro \href{https://en.wikipedia.org/wiki/Moneyball}{Moneyball}, el cual tuvo una gran 
influencia también fuera de Estados Unidos. Trata sobre cómo
un equipo de béisbol de bajo presupuesto se convirtió en uno de los mejores, usando estadísticas para reclutar a jugadores 
con habilidades hasta entonces minusvaloradas, como el \href{https://en.wikipedia.org/wiki/On-base_percentage}{porcentaje de veces que un bateador llega a una base}, o 
la \href{https://en.wikipedia.org/wiki/Slugging_percentage}{productividad de bateo} \cite{moneyball-ev}. El lanzamiento del 
libro fomentó que se generaran \href{https://thesportjournal.org/article/an-examination-of-the-moneyball-theory-a-baseball-statistical-analysis/}{preguntas} en torno al empleo de datos en deportes de equipo. 

A su vez, en 2005 hubo muchos intentos de relacionar redes complejas y fútbol \cite{lee2005passes}, 
e incluso en el 2004 se lanzó un desafío en la lista redes \cite{Bundio_Matías_2008} para predecir el resultado de la Eurocopa usando 
redes sociales. En cualquier caso, tanto el estallido del análisis de redes sociales como el libro contribuyeron 
a la expansión de este campo.

En los últimos años ha habido una revolución en el análisis de la organización 
y rendimiento de los equipos de fútbol y sus jugadores, por lo que es posible tener acceso a todos los 
eventos que ocurren en el campo, tales como pases, disparos o goles, todo ello con las coordenadas 
exactas de tiempo y posición, y el jugador responsable de cada evento. Por otro lado, también es 
posible llevar un registro de las posiciones de todos los jugadores en el campo, junto con el balón, 
lo que permite determinar la posición, velocidad y aceleración de cada jugador, y dando información 
muy útil sobre su desempeño físico y táctico. 

No obstante, los avances más determinantes e importantes tienen que ver con la posibilidad de aplicar o 
definir nuevos métodos y herramientas. En ese sentido, se pueden entender los roles de los jugadores 
como un todo, no solo como componentes aisladas sin interacciones entre ellos. Consecuentemente, es 
posible contruir redes de pases compuestas por nodos, correspondientes a los jugadores, y enlaces, 
que representan los pases entre ellos. Esta organización está además lejos de ser 
aleatoria. El análisis de la evolución de las redes de pases ha mostrado que sus propiedades 
cambian continuamente a lo largo de un partido y que eventos importantes tales como goles 
pueden afectar la organización de la red \cite{spatial-and-temporal-entropies}.

Asimismo, durante la última década las redes bayesianas \ref{def:BN} se han popularizado en el campo 
de la inteligencia artificial, y hay numerosos estudios que buscan predecir los resultados 
de partidos de fútbol empleándolas, para lo que construyen diferentes modelos \cite{prediction-barcelona}.
En \cite{razali-2017} consiguen una precisión predictiva del 75.09\%. \cite{dolores} introduce el uso 
de calificaciones dinámicas \ref{subsect:ratings} y redes bayesianas híbridas \ref{def:hybrid_BN} para hacer una predicción del resultado de un 
partido entre dos equipos \textit{a} y \textit{b} a partir de datos históricos de partidos 
en los que no participan ni \textit{a} ni \textit{b}. Ello nos muestra 
el increíble potencial que tienen estas redes también, y no solo las más tradicionales 
redes complejas \ref{def:CN} \cite{caldarelli2007scale}. 

A diferencia de los sistemas simples, muy pocas metodologías tratan la evaluación de la confiabilidad de
sistemas complejos, especialmente los configurados como redes, donde es difícil tomar en
consideración los diferentes vínculos y factores que pueden afectar la disponibilidad y confiabilidad de tales
sistemas En este contexto, las redes bayesianas permiten el
modelado de sistemas configurados como red y el cálculo de probabilidades marginales de la
nodos del sistema utilizando probabilidades previas y condicionales \cite{bn-and-cn}.

Adicionalmente, hay estudios \cite{smart-data} que ponen un gran énfasis 
en aplicar conocimiento causal al proceso de desarrollo del modelo, basándose 
en los datos que son necesarios para la predicción. De esa manera, consiguen predicciones precisas 
del cambiante rendimiento de equipos de fútbol. El modelo permite predecir, antes de que empiece 
una temporada, los puntos totales en la liga que se espera un equipo acumule a lo largo de la temporada, lo 
que supone un cambio de perspectiva con respecto a los artículos mencionados con anterioridad, y en su 
metodología construye y trabaja sobre la literatura anterior, extendiéndola.

Por otro lado, hay vertientes que hacen análisis de fútbol desde la inferencia estadística, desde 
el \textit{machine learning}, o ambos \cite{ML-inference}. En este \textit{paper}, muestran que 
los equipos están caracterizados por dónde en el campo realizan pases, y se 
pueden identificar por la manera en que pasan el balón. Usando mapas de calor de las localizaciones 
de los pases, consiguen una precisión del 87\% en una clasificación de veinte equipos. Emplean 
además la localización de los pases a lo largo de una posesión del balón para predecir tiros a 
puerta. Finalmente, usan los pesos del modelo de predicción para categorizar a los jugadores 
según el valor de sus pases. Nos muestra una vez más la diversidad que existe en el estudio 
y análisis del fútbol; no hay límites en cuanto a lo que se puede considerar y obtener.

\cite{entropy-analysis} afirma que para optimizar su rendimiento, un 
equipo debe mantener un equilibrio entre orden u organización, el cual 
propicia la cooperación entre sus miembros, y desorden o desorganización, que confunde 
al oponente y favorece el mantener un cierto grado de libertad. Para medir, usan entropía de 
Tsallis \ref{def:tsallis_entropy} a partir de los pases entre los jugadores, y concluyen que la posición de un equipo 
al final de la temporada está correlada con la entropía del mismo. 

Más específicamente, saber quién pasa el balón a quién significa reducir la entropía \ref{def:entropy} de los pases y maximizar 
la comunicación y cooperación entre los jugadores del equipo. A su vez, esto implicaría limitar sus 
grados de libertad al mover la pelota y sorprender al oponente, lo que expondría al equipo a contraataques. El precio 
pues de maximizar la certeza de los pases puede ser un comportamiento predecible y hacer al equipo vulnerable. Lo más importante 
de este \textit{paper} es que reivindica que el rendimiento de un equipo de fútbol puede 
ser modelado usando la entropía de sus pases de balón.

Y no solo eso, \cite{spatial-and-temporal-entropies} cuantifica la entropía espacial y temporal de 
equipos de fútbol y sus jugadores, también mediante interacciones basadas en pases. Sus resultados 
muestran que la entropía espacial cambia de acuerdo con la posición de los jugadores en el campo, y la 
organización de las redes de pases cambian a lo largo de un partido. 

\cite{network-analysis} fue uno de los primeros en investigar la entropía entre los jugadores de un equipo de 
fútbol. Analizan el desempeño y estilo de juego de la selección española en los mundiales del 2010 desde una 
perspectiva temporal, con lo que medidas globales como el número de pases consecutivos o el número de 
pases por minuto reflejan el éxito de un equipo en imponer su forma de juego.
Así, España consigue tener mayor posesión del balón y pases completados, lo que es una buena estrategia defensiva 
al privar al otro equipo de influenciar en el juego. Esto en último lugar determina el resultado final del partido, 
si bien es discutible la necesidad de tener tanto tiempo el balón, sin muchas veces avanzar a puerta. En el 
último partido muestran que la precisión disminuye y el juego es menos elaborado, al introducirse 
más emoción y nervios.

Pero, ¿cómo medimos la calidad de un equipo? No hay muchos artículos aplicando redes causales a partidos de 
fútbol. \cite{cerqueira} muestra que las estrategias ofensivas son más influyentes que las defensivas.



 