%%%%%%%%%%%%%%%%%%%%%%%%%%%%%%%%%%%%%%%%%%%%%%%%%%
%%%%%%%%%%%%%%%%%%%%%%%%%%%%%%%%%%%%%%%%%%%%%%%%%%
%%
%% Based one the "beamer-greek-two" template provided 
%% by the Laboratory of Computational Mathematics, 
%% Mathematical Software and Digital Typography, 
%% Department of Mathematics, University of the Aegean
%% (http://myria.math.aegean.gr/labs/dt/)
%%
%% Adapted by John Liaperdos, October-November 2014
%% (ioannis.liaperdos@gmail.com)
%%
%% Last update: 22/06/2017 (English Support)
%% Source: https://es.overleaf.com/latex/templates/thesis-presentation-template-beamer-english-version-dpt-of-computer-engineering-technological-educational-institute-of-peloponnese/vwhtyshhtqmg
%%%%%%%%%%%%%%%%%%%%%%%%%%%%%%%%%%%%%%%%%%%%%%%%%%
%%%%%%%%%%%%%%%%%%%%%%%%%%%%%%%%%%%%%%%%%%%%%%%%%%
%%
\PassOptionsToPackage{unicode}{hyperref}
\PassOptionsToPackage{naturalnames}{hyperref}
\documentclass{beamer} 
\usepackage{babel}
\usepackage[utf8]{inputenc}


%%% FONT SELECTION %%%%%%%%%%%%%%%%%
%%% we choose a sans font %%%%%%%%%%
\usepackage{kmath,kerkis} 
%\usepackage[default]{gfsneohellenic} 
%%%%%%%%%%%%%%%%%%%%%%%%%%%%%%%%%%%%

\usepackage{color}
\usepackage{amsmath}
\usepackage{amssymb}

\usepackage{epstopdf}
\usepackage{graphicx}
\graphicspath{{./img/}}

%%
% load TEI-Pel - specific layout
\usepackage{TeiPel_En_Beamer_Layout}
\setTeipelLayout{draft,newlogo}% options: "draft", "newlogo"

%%%%%%%%%%%%%%%%%%%%%%%%%%%%%%%%%%%%%%%%%%%%%%%%%%%%%%%%%%%%
% Thesis Info %%%%%%%%%%%%%%%%%%%%%%%%%%%%%%%%%%%%%%%%%%%%%%
%%%%%%%%%%%%%%%%%%%%%%%%%%%%%%%%%%%%%%%%%%%%%%%%%%%%%%%%%%%%
	% title
		\title[Análisis de redes causales en deportes de equipo]{Análisis de redes causales en deportes de equipo}	
	% author 
    % (In the mandatory argument "{}", separate multiple
    % authors with "\and" - use "\\" for better author name formatting
    % in the title page. In the optional argument "[]" include all
	% author names, with no "\and" or text formatting macros.)
	% Example: 
    %\author[A. Author Albert Einstein]{Anthony Author \and Albert Einstein}
		\author[E. Merelo]{Elena Merelo}
	% supervisor	
		% \supervisor{Supervisor}{Mister Supervisor}{Professor}
	% date
		\presentationDate{7 de septiembre de 2022}
%%%%%%%%%%%%%%%%

\begin{document}

% typeset front slides
	\typesetFrontSlides

%%%%%%%%%%%%%%%%
% Your Slides Start here:

%%%%
\section{Introducción}

%%
\subsection[Problema]{Descripción del problema}

\begin{frame}{Motivación}
	\framesubtitle{El fútbol es un deporte de equipo \textit{complicado}}
    Como parte del cuerpo técnico de un equipo habrá que decidir:
    \begin{itemize}
		\item Quiénes jugarán
		\pause
		\item Dónde se posicionarán
		\pause
        \item Cuándo y qué cambios habrá
		\pause
        \item Alineación por la que se abogará
        \pause
        \item Así como a quién fichar, a quién vender, qué hacer en los entrenamientos 
        \pause
        \item Entran muchas variables \textit{en juego}
	\end{itemize}
\end{frame}

\begin{frame}{Objetivos del trabajo}
	\begin{itemize}
		\item <1->Entender cómo se pueden aplicar técnicas estadísticas en deportes de equipo, 
        y adaptar resultados en ese campo a los tipos de datos que existen aquí.
		\item <2->Crear una herramienta para que personas del equipo técnico de un equipo
        de fútbol, como analistas de datos, analistas técnicos, analistas de rendimiento físico o 
        entrenadores, puedan tomar decisiones en base al estudio que se haga antes, durante o después de 
        un partido.
	\end{itemize}
    El repositorio en Github del proyecto se puede encontrar \href{https://bit.ly/memoriaTFG_emerelo}{aquí}.
\end{frame}

\begin{frame}{Descripción}
	\begin{itemize}
		\item <1->¿La entropía mejora las posibilidades de marcar gol? 
        \item <2->Si cambia la entropía del equipo, ¿podemos determinar la causa?
        \item <3->¿Es posible ver la entropía reflejada en un tipo de visualización de las redes de pases?
        \item <4->¿Hasta qué punto es determinante la entropía a nivel de jugador o equipo?
	\end{itemize}
\end{frame}

\begin{frame}{Metodología}
	\begin{itemize}
		\item \textbf{Desarrollo ágil}
		\pause
		\item \textit{Design thinking}
	\end{itemize}
\end{frame}

%%
\subsection{Definiciones}
\begin{frame}{Redes bayesianas}
    \begin{definition}[Redes bayesianas] 
        Una red bayesiana $\mathfrak{B} = \lbrace \mathcal{G}, \mathbb{P} \rbrace$ está definida por:
        \begin{itemize}
            \item Un grafo dirigido acíclico $\mathcal{G}=(V,E)$ donde $V$ es un conjunto de nodos y $E$ 
            es un conjunto de aristas.
            \item Un espacio de probabilidad $(\Omega, \mathbb{P})$.
            \item Un conjunto de variables aleatorias $\textbf{V}=V[i], i=1,...,N$ con N lista de variables aleatorias, 
            asociadas con los nodos del grafo $(\Omega, \mathbb{P})$ 
            de tal manera que $\mathbb{P}(V[1],...,V[N])= \prod_{i=1}^{n}\mathbb{P}(V[i]|pa(V[i]))$, donde $pa(V[i])$ es el 
            conjunto de los nodos padre de $V[i]$ en $\mathcal{G}$.  
        \end{itemize}
    \end{definition}     
\end{frame}

\begin{frame}{Redes bayesianas}
	\begin{center}
		Ejemplo de grafo dirigido acíclico \\[12pt]
		\includegraphics[width=0.35\textwidth,keepaspectratio]{dag.png}
		\\
		\footnotesize(Una red bayesiana consiste en un modelo estructural y un conjunto de probabilidades)
    \end{center}
\end{frame}


\begin{frame}{Redes causales}
    \begin{block}
        <1->{}
            \begin{itemize}
                \item Red bayesiana en la que los padres de cada nodo son sus causas directas.
                \item Satisface la {\em condición de Markov causal}
            \end{itemize}
    \end{block}
    \begin{exampleblock}
        <2->{Condición de Markov causal}
            \begin{itemize}
                \item Dadas las causas directas, el fenómeno asociado a un nodo es independiente de 
                los que no tienen efecto sobre él. Esta asunción permite que la distribución 
                conjunta de las variables en una red causal sea factorizada como :
                $$ P(X_{1}, \dots,X_{N}) = \prod_{i=1}^{N} P(X_{i} | pa(X_{i}))$$ 
            \end{itemize}
    \end{exampleblock}
\end{frame}

\begin{frame}{Entropía}
	\begin{definition}[Entropía] \label{def:entropy}
        Para una variable aleatoria discreta $X$ definimos su entropía como:  
        
        $$H(X):= - \sum_{x} P(x)\log[P(x)]$$
        
        bits, donde $X$ toma valores en $\mathcal{X}$, $P:\mathcal{X} \rightarrow [0,1]$, $P(x)$ es la 
        probabilidad de que $X$ esté en el estado $x$. La 
        entropía conjunta de las variables $X_1...X_N$ se define por 
        $$H(X_1,...,X_N)=-\sum_{x_1}...\sum_{x_N}P(x_1,...,x_N)\log[P(x_1,...,x_N)]$$
    \end{definition}
\end{frame}

%%%%
\section{Contribuciones}

%%
\subsection{Implementación}

\begin{frame}{Summary}
   	\begin{alertblock}{Attention}
   		\textlatin{This is an important alert}
   	\end{alertblock}
\end{frame}

%
\subsection{Resultados principales}

\begin{frame}{Summary}
	\begin{itemize}
		\item The \textcolor{red}{first main message} of our talk.
		\item The \textcolor{red}{second main message} of our talk.
		\item Maybe a \textcolor{red}{third message}, but ... no more.
	\end{itemize}
	\vskip0pt plus.5fill
	\begin{itemize}
		\item Conclusion.
	\end{itemize}
	\begin{itemize}
		\item Future work.
		\item Discussion.
	\end{itemize}
\end{frame}

\subsection{Conclusiones y trabajos futuros}

\begin{frame}{Summary}
	\begin{itemize}
		\item The \textcolor{red}{first main message} of our talk.
		\item The \textcolor{red}{second main message} of our talk.
		\item Maybe a \textcolor{red}{third message}, but ... no more.
	\end{itemize}
	\vskip0pt plus.5fill
	\begin{itemize}
		\item Conclusion.
	\end{itemize}
	\begin{itemize}
		\item Future work.
		\item Discussion.
	\end{itemize}
\end{frame}

\begin{frame}{Bibliografía}
	\begin{thebibliography}{2}
		\beamertemplatebookbibitems
		\bibitem{Author1990}A.\ Author. \newblock\emph{Handbook of Everything}.\newblock
\textlatin{Some Press, \oldstylenums{1990}}.

		\beamertemplatearticlebibitems
		\bibitem{Someone2002}B.\ Author.\newblock On this and that\emph{.}
\newblock\emph{Journal on This and That}. 
\oldstylenums{2}(\oldstylenums{1}):\oldstylenums{50}--\oldstylenums{100}, 
\oldstylenums{2000}.
	\end{thebibliography}
\end{frame}

%%
\end{document}